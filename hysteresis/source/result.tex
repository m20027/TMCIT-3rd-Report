\clearpage

\section{結果}
\begin{itemize}
	\item 入力電圧-励磁電流-電力-電力の関係は\wtab{re1}のようになった.入力電圧上昇とともに電流,電力,位相の増加が確認できる.
	これは電力が電圧,電流に比例するためである.
	\begin{table}[h]
	\centering
	\caption{励磁電流特性}
	\label{tab:re1}
	\begin{tabular}{cccc}
	\hline
	入力電圧$V\,[\rm{V}]$& 電流$i_{0}\,[\rm{A}]$& 電力$P_{0}\,[\rm{W}]$& 位相\,[\rm{deg}]  \\ 
	\hline
	75  & 0.307    & 1.84     & 85.42 \\
	80  & 0.337    & 2.10      & 85.53 \\
	85  & 0.378    & 2.36     & 85.79 \\
	90  & 0.429    & 2.64     & 86.08 \\
	95  & 0.483    & 2.96     & 86.30 \\
	100 & 0.552    & 3.34     & 86.53 \\ \hline
	\end{tabular}
	\end{table}
	\item 入力電圧が$80\,[\rm{V}], 100\,[\rm{V}]$の励磁電流をそれぞれ\wfig{1}に示す.
	\begin{figure}[h]
	\centering
	\includegraphics[scale=0.8]{./data/graph/1.pdf}
	\caption{励磁電流}
	\label{fig:1}
	\end{figure}
	\item 入力電圧が$80\,[\rm{V}], 100\,[\rm{V}]$のヒステリシスループをそれぞれ\wfig{}に示す.
	\item ヒステリシスループを用いて作図した励磁電流を\wfig{}
\end{itemize}