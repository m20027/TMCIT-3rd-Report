\documentclass[dvipdfmx]{ujarticle}
\usepackage{eee}

\begin{document}
\title{平成28年 電磁気学II}
\date{}
\author{大山主朗}

\maketitle

\section*{平成28年 電磁気学II 第1回小テスト}
\section{以下の(a)及び(c)に示す物理定数は電磁気学を修めた者であれば常識的に覚えていなければならない数値である.それぞれの値を示せ.}
\begin{enumerate}[(a)]
	\item 真空の誘電率$\varepsilon_{0}:8.854 \times 10^{-12}\,\rm{F/m}$
	\item 真空の透磁率$\mu_{0}:1.257 \times 10^{-6}\,\rm{H/m}$
	\item 電子の電荷$e:-1.602 \times 10^{-19}\,\rm{C}$
\end{enumerate}

\section{$xyz$直交座標系における$x, y, z$方向の基本ベクトルを$\boldsymbol{i}, \boldsymbol{j}, \boldsymbol{k}$とする。$\boldsymbol{a}=a_{x}\boldsymbol{i}+a_{y}\boldsymbol{j}+a_{z}\boldsymbol{k}, \boldsymbol{b}=b_{x}\boldsymbol{i}+ b_{y}\boldsymbol{j}+ b_{z}\boldsymbol{k}$なる二つのベクトルがある.このとき以下の各式を計算せよ.}
\begin{enumerate}[(a)]
	\item $\boldsymbol{a}+\boldsymbol{a}=(a_{x}+b_{x})\boldsymbol{i}+(a_{y}+b_{y})\boldsymbol{j}+(a_{z}+b_{z})\boldsymbol{k}$
	\item $\boldsymbol{a}-\boldsymbol{b}=(a_{x}-b_{x})\boldsymbol{i}+(a_{y}-b_{y})\boldsymbol{j}+(a_{z}-b_{z})\boldsymbol{k}$
	\item $a=|\boldsymbol{a}|=\sqrt{a_{x}^{2}+a_{y}^{2}+a_{z}^{2}}$
	\item $\boldsymbol{a}$方向の単位ベクトル$a_{e}=\frac{a}{|\boldsymbol{a}|}=\frac{a}{\sqrt{a_{x}^{2}+a_{y}^{2}+a_{z}^{2}}}$
	\item $\boldsymbol{a}\cdot \boldsymbol{b}=a_{x}b_{x}+a_{y}b_{y}+a_{z}b_{z}$
	\item $\boldsymbol{a}\cdot \boldsymbol{a}=a_{x}^{2}+a_{y}^{2}+a_{z}^{2}$
	\item $\frac{1}{\boldsymbol{a}}=\frac{\boldsymbol{a}}{a^{2}}=\frac{\boldsymbol{a}}{a_{x}^{2}+a_{y}^{2}+a_{z}^{2}}=\frac{1}{a_{x}^{2}+a_{y}^{2}+a_{z}^{2}}(a_{x}\boldsymbol{i}+a_{y}\boldsymbol{j}+a_{z}\boldsymbol{k})$
	\item $\boldsymbol{a}\times \boldsymbol{b}=(a_{y}b_{z}-a_{z}b_{y})\boldsymbol{i}+(a_{z}b_{x}-a_{x}b_{z})\boldsymbol{j}+(a_{x}b_{y}-a_{y}b_{x})\boldsymbol{k}$
	\item $|\boldsymbol{a}\times \boldsymbol{b}|=\sqrt{(a_{y}b_{z}-a_{z}b_{y})^{2}+(a_{z}b_{x}-a_{x}b_{z})^{2}+(a_{x}b_{y}-a_{y}b_{x})^{2}}$
	\item $\boldsymbol{b}\times \boldsymbol{a}=(a_{z}b_{y}-a_{y}b_{z})\boldsymbol{i}+(a_{x}b_{z}-a_{z}b_{x})\boldsymbol{j}+(a_{y}b_{x}-a_{x}b_{y})\boldsymbol{k}$
	\item $\boldsymbol{a}, \boldsymbol{b}$の間の角を$\theta$としたときの$\sin \theta=\frac{|\boldsymbol{a}\times \boldsymbol{b}|}{|\boldsymbol{a}||\boldsymbol{b}|}=\sqrt{\frac{(a_{y}b_{z}-a_{z}b_{y})^{2}+(a_{z}b_{x}-a_{x}b_{z})^{2}+(a_{x}b_{y}-a_{y}b_{x})^{2}}{(a_{x}^{2}+a_{y}^{2}+a_{z}^{2})(b_{x}^{2}+b_{y}^{2}+b_{z}^{2})}}$
	\item $\boldsymbol{a}, \boldsymbol{b}$の間の角を$\theta$としたときの$\cos \theta=1-\sin^{2}\theta=1-\left(\frac{(a_{y}b_{z}-a_{z}b_{y})^{2}+(a_{z}b_{x}-a_{x}b_{z})^{2}+(a_{x}b_{y}-a_{y}b_{x})^{2}}{(a_{x}^{2}+a_{y}^{2}+a_{z}^{2})(b_{x}^{2}+b_{y}^{2}+b_{z}^{2})}\right)$
\end{enumerate}

\section{直線状電流$I$が流れている.このとき,電流の周囲に発生する磁束密度$\boldsymbol{B}$を求めよ。特に$\boldsymbol{B}$の大きさと方向を明示せよ.}
\begin{align*}
水平&方向をx軸,鉛直上向きをy軸,手前側にz軸をとる\\
今,&直線状電流(上端のy座標:y_{b},下端のy座標:y_{a})からx=a離れた地点の磁界Hを考える\\
ここで&x軸とy_{b}を通る\boldsymbol{r}がなす角を\beta -\frac{\pi}{2},x軸とy_{a}を通る\boldsymbol{r}がなす角を-\left(\frac{\pi}{2}-\alpha \right)とする\\
\boldsymbol{i}&,\boldsymbol{j},\boldsymbol{k}をx,y,z方向の基底ベクトルとする\\
d\boldsymbol{l}&=dy\boldsymbol{j}, \quad \boldsymbol{r}=a\boldsymbol{i}-y\boldsymbol{j}, \quad r=\sqrt{a^{2}+y^{2}}\\
d\boldsymbol{l}\times \boldsymbol{r}&=
\begin{vmatrix}
\boldsymbol{i} & \boldsymbol{j} & \boldsymbol{k}\\
0 & dy & 0\\
a & -y & 0 
\end{vmatrix}
=dy
\begin{vmatrix}
\boldsymbol{i} & \boldsymbol{k} \\
a & 0
\end{vmatrix}
=-a dy \boldsymbol{k}\\
d\boldsymbol{H}&=\frac{Id\boldsymbol{l}\times \boldsymbol{r}}{4 \pi r^{3}}\\
\boldsymbol{H}&=\int _{C} d\boldsymbol{H}\\
&=-\frac{a\boldsymbol{k}I}{4\pi}\int_{y_{a}}^{y_{b}} (a^{2}+y^{2})^{-\frac{3}{2}}\,dy\\
y&=a\tan \theta と置換する.\therefore dy=\frac{a}{\cos \theta }d \theta, \quad 積分範囲は\alpha -\frac{\pi}{2} \to \beta -\frac{\pi}{2}に変化する\\ 
\boldsymbol{H}&=-\frac{a\boldsymbol{k}I}{4\pi}\frac{1}{a^{3}}\int_{\alpha-\frac{\pi}{2}}^{\beta -\frac{\pi}{2}} (1+\tan^{2} \theta )^{-\frac{3}{2}} \frac{a}{\cos ^{2} \theta}\,d \theta\\
&=-\frac{\boldsymbol{k}I}{4\pi a} \int_{\alpha -\frac{\pi}{2}}^{\beta -\frac{\pi}{2}} \cos \theta d\theta\\
&=-\frac{\boldsymbol{k}I}{4\pi a}\left\{ \sin \left(\beta -\frac{\pi}{2}\right) - \sin \left(\alpha -\frac{\pi}{2}\right) \right\}\\
&=-\frac{\boldsymbol{k}I}{4 \pi a} \left(-\cos \beta +\cos \alpha \right)\\
&=\frac{\boldsymbol{k}I}{4\pi a} (\cos \alpha -\cos \beta)\\
ここで&線電流の長さが無限より,\alpha =0,\beta =\infty であるため\\
H&=\frac{I}{2\pi a}\,[\rm{A/m}],z軸逆向き.
\end{align*}

\section{一様な磁束密度$\boldsymbol{B}$の中に長さ$s$の直線電流$I$が流れている.磁束密度$\boldsymbol{B}$と電流$I$のなす角が$\frac{\pi}{2}$のとき,電流$I$にはたらく力$\boldsymbol{F}$を求めよ.}
\begin{align*}
	|d\boldsymbol{F}|&=|Id\boldsymbol{l}\times \boldsymbol{B}|\\
	&=IB\sin \frac{\pi}{2}\\
	&=IB\\
	|\boldsymbol{F}|&=\int_{0}^{s} |d\boldsymbol{F}|\\
	&=sIB\,[\rm{N}]
\end{align*}
\section{一様な磁束密度$\boldsymbol{B}$の中に長さ$s$の直線電流$I$が流れている.磁束密度$\boldsymbol{B}$と電流$I$のなす角が$\frac{\pi}{6}$のとき,電流$I$にはたらく力$\boldsymbol{F}$を求めよ.}
\begin{align*}
	|d\boldsymbol{F}|&=|Id\boldsymbol{l}\times \boldsymbol{B}|\\
	&=IB\sin \frac{\pi}{6}\\
	&=\frac{IB}{2}\\
	|\boldsymbol{F}|&=\int_{0}^{s} |d\boldsymbol{F}|\\
	&=\frac{sIB}{2}\,[\rm{N}]
\end{align*}

\section{一様な磁束密度$\boldsymbol{B}$の中に質量$m$,電荷$q$の電荷粒子が$\boldsymbol{B}$に垂直に初速度$v_{0}$で飛び込んで円運動をする.このとき以下の角値を求めよ.}
\begin{enumerate}[(a)]
	\item 電荷粒子にはたらくローレンツ力$F$
	\begin{align*}
		F&=qv_{0}B\sin \theta \\
		&=qv_{0}B\,[\rm{N}]
	\end{align*}
	\item 電荷粒子にはたらく遠心力$F$
	\begin{align*}
		F&=\frac{mv_{0}^{2}}{r}\,[\rm{N}]
	\end{align*}
	\item 円運動の回転半径$r$
	\begin{align*}
		qv_{0}B&=\frac{mv_{0}^{2}}{r}\\
		r&=\frac{mv_{0}}{qB}\,[\rm{m}]
	\end{align*}
	\item 円運動の周期$T$
	\begin{align*}
		T&=\frac{2\pi r}{v_{0}}\,[\rm{s}]
	\end{align*}
	\item 円運動の角速度$\omega$
	\begin{align*}
		\omega &=\frac{2\pi}{T}\\
		&=\frac{v_{0}}{r}\,[\rm{rad/s}]
	\end{align*}
\end{enumerate}

\clearpage
\setcounter{section}{0}
\section*{平成28年 電磁気学II 前期中間試験}
\section{以下の(a)及び(c)に示す物理定数は電磁気学を修めた者であれば常識的に覚えていなければならない数値である.それぞれの値を示せ.}
\begin{enumerate}[(a)]
	\item 真空の誘電率$\varepsilon_{0}:8.854 \times 10^{-12}\,\rm{F/m}$
	\item 真空の透磁率$\mu_{0}:1.257 \times 10^{-6}\,\rm{H/m}$
	\item 電子の電荷$e:-1.602 \times 10^{-19}\,\rm{C}$
\end{enumerate}

\section{$xyz$直角座標空間において,$xy$平面に中心が原点O,半径$a$の円状に電流$I$が流れている.このとき,原点Oにおける磁束密度$\boldsymbol{B}$を求めよ.}
\begin{align*}
\end{align*}

\section{$xyz$直角座標空間において,$xy$平面に中心が原点O,半径$a$の円状に電流$I$が流れている.このとき,原点Oにおける磁束密度$\boldsymbol{B}$を求めよ.}
\begin{align*}
\end{align*}

\section{}
\begin{align*}
\end{align*}

\section{}
\begin{align*}
\end{align*}

\section{}
\begin{align*}
\end{align*}

\section{}
\begin{align*}
\end{align*}

\clearpage
\setcounter{section}{0}
\section*{平成28年 電磁気学II 前期第2回小テスト}
\section{以下の(a)及び(c)に示す物理定数は電磁気学を修めた者であれば常識的に覚えていなければならない数値である.それぞれの値を示せ.}
\begin{enumerate}[(a)]
	\item 真空の誘電率$\varepsilon_{0}:8.854 \times 10^{-12}\,\rm{F/m}$
	\item 真空の透磁率$\mu_{0}:1.257 \times 10^{-6}\,\rm{H/m}$
	\item 電子の電荷$e:-1.602 \times 10^{-19}\,\rm{C}$
\end{enumerate}

\clearpage
\setcounter{section}{0}
\section*{平成28年 電磁気学II 前期期末試験}
\section{以下の(a)及び(c)に示す物理定数は電磁気学を修めた者であれば常識的に覚えていなければならない数値である.それぞれの値を示せ.}
\begin{enumerate}[(a)]
	\item 真空の誘電率$\varepsilon_{0}:8.854 \times 10^{-12}\,\rm{F/m}$
	\item 真空の透磁率$\mu_{0}:1.257 \times 10^{-6}\,\rm{H/m}$
	\item 電子の電荷$e:-1.602 \times 10^{-19}\,\rm{C}$
\end{enumerate}

\section{単位長さ当たり$n$巻の無限長ソレノイドコイルに電流$I$を流した.この
とき無限長ソレノイドコイルがコイル周辺に作る磁束密度$\boldsymbol{B}$を求めよ.}
\begin{align*}
\end{align*}

\section{円環中心線の半径を$R$,円形断面の半径をr,巻き数$N$の環状ソレノイ
ドコイルに電流$I$を流したとき,コイル内の磁束密度$\boldsymbol{B}$を求めよ}
\begin{align*}
\end{align*}

\section{強磁性体,常磁性体,反磁性体の各磁性体の特徴を,比透磁率および磁化率という言葉を用いて説明せよ.}

\section{磁化されていない強磁性体に磁界$H$を外部から印加し,強磁性体内部での磁束密度$B$を観測すると,図3に示すような結果が得られた.このとき,図中の行程1:点$\rm{O}\to$点$\rm{P_{1}}$,行程 2:点$\rm{P_{1}}\to$点$\rm{P_{2}}$,行程3:点$\rm{P_{2}}\to$点$\rm{P_{3}}$,行程4:点$\rm{P_{3}}\to$点P4,行程5:点P4 $\to$点$\rm{P_{5}}$, 行程6:点$\rm{P_{5}}\to$点P6,行程7:点$\rm{P_{6}}\to$点P1の7つの行程に着目して,測定結果を説明せよ.}
\end{document}
