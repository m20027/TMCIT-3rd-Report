\documentclass[11pt,dvipdfmx]{ujarticle}
\usepackage{eee,scalefnt,graphicx}
\bibliography{3rd_ele-EE_Measurement}

\newcounter{mycounter} 
\setcounter{mycounter}{0} 
\newcommand{\useMycounter}[1][]{\refstepcounter{mycounter}{#1}プログラム{\themycounter}: }

\begin{document}

\begin{jikkenTitle}
 \gakunen{3} 
 \numTitle{1}{仮想計測器の利用と基礎的な数値処理} 
 \subTitle{(Use of Virtual Instruments and basic numeric processing)} 
 \jikkenbi{令和04年05月12日(木)} 
 \jikkenbiII{令和04年05月19日(木)} 
 \kyoudou{3326 塚原 秀翔}
 \yoteibi{05/19}
 \yoteibiII{05/26}
 \yoteibiIII{06/02}
 \hanNumberName{1}{3309}{大山 主朗}
\end{jikkenTitle}

\section{目的}
本実験では,
\begin{itemize}
	\item LabVIEWおよびMyRIOを使用して,素子の電圧電流特性について自動計測の方法を習得する.
	\item 測定データから近似直線式の傾き,切片を求める計算方法を習得する.
	\item 電圧電流特性から抵抗値を求める方法について習得する.
	\item 実験による,センサ等の電流電圧特性を確認する
\end{itemize}
ことを目的とする.

\clearpage

\section{原理}
\subsection{磁束(magnetic flux)\cite{113028227042}}
磁気量$q_{m}$の磁極からは$q_{m}$本の磁束が発生し,磁束は途切れたり枝分かれすることはない.
磁束に垂直な単位面積の面を貫く磁束を磁束密度(magnetic flux density)といい,磁束密度$\boldsymbol{B}$と磁界$\boldsymbol{H}$には
\begin{equation}
	\boldsymbol{B}=\mu\boldsymbol{H}\,[\rm{Wb/m^{2}}] \,or\, [\rm{T}] 
\end{equation}
という関係がある.

\subsection{磁気双極子(magnetic dipole)\cite{7652}}
正負の磁極の対.本質的には微小円電流で以下のように表される.
\begin{equation}
	\boldsymbol{m}=\mu_{0}IS\boldsymbol{n}
\end{equation}
ここで,$S$はループ面の面積,$\boldsymbol{n}$はループ面の単位法線ベクトルである.

\subsection{磁気モーメント(magnetic moment)\cite{7652}\cite{113028227162}}
磁気モーメントは以下のように表すことができる.
\begin{equation}
	\boldsymbol{\mu}=IS\boldsymbol{n}
\end{equation}

\subsection{磁化(magnetization)\cite{11302042}}
磁界に対して応答を示す物質を磁性体(magnetic material)という.
通常の磁性体には多数の磁気双極子が含まれている.単位体積に含まれる磁気モーメントの和を磁化と呼ぶ.
ここで薄い板状の磁性体を考える.板に垂直方向に一様な磁界がかかっており,外部における磁場を$H_{0}$とすると,磁性体の外部での磁束密度$B$は
\begin{equation}	
	B=\mu_{0}H_{0}\,[\rm{T}]
\end{equation}
という関係を満たす.一方,この場合には磁性体は板の厚み方向に磁化されており,その値を$M$とすると,磁性体の両面には単位面積あたり$\pm \mu_{0}M$の磁極が発生する.このとき磁極は磁性体内部に
$-M$という磁界を発生させる.これらより磁性体内部の磁束密度以下のようになる.
\begin{align}
	H&=H_{0}-M\nonumber\\
	\mu_{0}H&=\mu_{0}H_{0}-\mu_{0}M\nonumber\\
	\mu_{0}H&=B-\mu_{0}M\nonumber\\
	B&=\mu_{0}(H+M)\,[\rm{T}]
\end{align}
となる.向きも含めて考えると上式は\weq{mo}のようになる.
\begin{equation}
	\boldsymbol{B}=\mu_{0}(\boldsymbol{H}+\boldsymbol{M})\,[\rm{T}]
	\label{eq:mo}
\end{equation}
また,磁界と磁化の関係を\weq{mmo}のように表現する場合
\begin{equation}
	\boldsymbol{M}=\chi\boldsymbol{H}\,[\rm{T}]
	\label{eq:mmo}
\end{equation}
$\chi$を磁化率(単位は無次元)などと呼び,\weq{mo}に代入し,\weq{bh}の関係を用いると
\begin{equation}
	\boldsymbol{B}=\mu \boldsymbol{H}\,[\rm{T}]
	\label{eq:bh}
\end{equation}
\weq{mu}のような関係があることがわかる.
\begin{align}
	\boldsymbol{B}&=\mu_{0}(\boldsymbol{H}+\boldsymbol{M})\nonumber\\
	\mu \boldsymbol{H}&=\mu_{0}(\boldsymbol{H}+\chi\boldsymbol{H})\nonumber\\
	\mu &=\mu_{0}(1+\chi)\,[\rm{F/m}]\label{eq:mu}
\end{align}

\subsection{キュリーの法則(Curie's law)\cite{76972}}
常磁性物質において,磁化率と温度の関係(反比例)を示す法則で以下のように表すことができる.
\begin{equation}
	\chi=\frac{C}{T}\,[\rm{-}]
\end{equation}
ここで,$C$はキュリー定数$\,[\rm{K}]$,$T$は絶対温度$\,[\rm{K}]$

\subsection{磁気回路(magnetic circuit)}
\wfig{hys:jikikairo}に示すように断面積$S\,[\mathrm{m}^2]$,平均磁路長$L\,[\mathrm{m}]$の鉄心に巻数$N_1\,[\mathrm{Turn}]$のコイルを巻き,これに$I\,[\mathrm{A}]$の電流を流すと,起磁力$N_1\cdot I\,[\mathrm{A}\cdot\rm{Turn}]$を生じる.
この起磁力により
\begin{equation}
	\phi = \frac{N_1\cdot I}{R_m}
\end{equation}
の磁束$\phi\,[\mathrm{Wb}]$を生じる.ここで$R_m$は以下に示す磁気抵抗である.
\begin{equation}
	R_m= \frac{L}{\mu_0 \mu_s S}
\end{equation}
ただし,$\mu_0 = 4\pi\times 10^{-7}\,\mathrm{F/m}$ は真空の透磁率であり,$\mu_s$は鉄心の比透磁率である.
ここで,磁路1\,mあたりの起磁力を磁化力$H\,[\mathrm{A/m}]$という. 磁化力$H$は
\begin{equation}
	H=\frac{N_1\cdot I}{L}
\end{equation}
である.
\begin{figure}[htbp]
	\centering
	\includegraphics[width=70mm]{fig/magnetism_circuit.pdf}
	\caption{磁気回路}
	\label{fig:hys:jikikairo}
\end{figure}

鉄心の磁化力$H$と磁束密度$B$との関係を示す曲線をB-H曲線といい,一般に\wfig{hys:bhcurve}(a)のような飽和特性になる.
また磁化力$H$を正負の方向に増減すると,\wfig{hys:bhcurve}(b)の様なヒステリシス曲線(Hysteresis curve)になる.
\begin{figure}[htbp]
	\centering
	\begin{tabular}{cc}
		\includegraphics[width=70mm]{fig/bhcurve.pdf} &
		\includegraphics[width=70mm]{fig/hysteresis.pdf} \\
		(a) B-H 曲線 & (b) ヒステリシス曲線
	\end{tabular}
	\caption{B-H曲線とヒステリシス曲線}
	\label{fig:hys:bhcurve}
\end{figure}

\subsection{交流磁化特性}
\label{zika}
\wfig{hys:transformer}の変圧器のように,鉄心に巻かれた巻数$N_1$のコイルに交流電圧$V_1$を加えると,鉄心中に交番磁束$\dot{\phi}$を作るための電流(励磁電流)$i_0$が流れる.このとき磁束密度$B$と磁化力$H$との間にはヒステリシス特性があるため,励磁電流は\wfig{hys:hizumi}のようにひずみを生ずる.この現象を逆に利用して,励磁電流$i_0$と交番磁束$\dot{\phi}$の波形をなんらかの方法で取り出し,オシロスコープのX軸に励磁電流$i_0$の波形,Y軸に交番磁束$\dot{\phi}$の波形を入力すれば,オシロスコープの画面に鉄心のヒステリシス特性(B-H曲線)が描かれる.
\begin{figure}[htbp]
	\centering
	\includegraphics[width=140mm]{fig/transformer.pdf}
	\caption{変圧器の交流磁化特性測定回路}
	\label{fig:hys:transformer}
\end{figure}

励磁電流$i_0$の波形を直接取り出すのは難しいので,\wfig{hys:transformer}において励磁電流$i_0$が抵抗$R_h$を流れるときの電圧変化,すなわち
\begin{equation}
	V_h = i_0R_h
\end{equation}
として取り出す.また,交番磁束$\dot{\phi}$は次の様にして取り出す.

\wfig{hys:transformer}において二次巻線$N_2$と鎖交する磁束の時間に対する変化が二次誘起電圧$e_2$として現れるため
\begin{equation}
	e_2 = -N_2\frac{d \phi}{dt}
	\label{eq:hys:e2}
\end{equation}
となり,\weq{hys:e2}を変形すると
\begin{equation}
	d \phi = \frac{1}{N_2}\times e_2\times dt
	\label{eq:hys:dphi}
\end{equation}
となるから,交番磁束$\phi$は\weq{hys:dphi}を積分すれば求まることとなる.すなわち,二次巻線に発生する電圧$e_2$を時間で積分すればよい.そこで二次側にCR積分回路を接続しコンデンサCの両端から$e_2$を積分した,交番磁束に比例した電圧をとりだす.

\begin{figure}[h]
  \begin{minipage}[c]{0.5\hsize}
    \centering
    \includegraphics[scale=1.2]{fig/hizumi_a.pdf} 
    \caption{ヒステリシス現象のない場合}
  \end{minipage}\\
  \begin{minipage}[c]{0.5\hsize}
    \centering
    \includegraphics[scale=1.2]{fig/hizumi_b.pdf}
    \caption{ヒステリシス現象のある場合}
  \end{minipage}
  \centering
  \caption{ヒステリシス現象}
   \label{fig:hys:hizumi}
\end{figure}

\subsection{磁区(magnetic domain)\cite{7697152}}
\wfig{domain}にヒステリシスループと磁区の関係を示す.
ヒステリシスループにおいてループと縦軸が交わる場所での磁化値を残留磁化といい,ループと横軸が交わる場所での磁界の値を保磁力という.
強磁性体は外部磁界がなくても自発磁化を持っているが,全体が一様に磁化されていると外部の空間に磁界を発生し,その磁界によるエネルギーが余分に余ってしまう.
そのため,強磁性体は自らを磁化の向きが異なる区間を磁区といい,磁区どうしの境界面を磁壁という.
強磁性は磁気モーメントどうしにお互いに同じ向きを向うとする相互作用がはたらくことで出現する.磁壁の両側では異なる向きの磁気モーメントが対峙することになるので,磁壁では余分なエネルギーが発生する.
このエネルギーの増加と,外部に磁場を発生させることによるエネルギーの増加の合計を最も小さくするように磁区構造が決まる.
\begin{figure}[h]
	\centering
	\includegraphics[scale=0.35]{fig/domain.png}
	\caption{ヒステリシスループと磁区\cite{cite-keygsdfz}}
	\label{fig:domain}
\end{figure}

\subsection{磁気飽和現象\cite{xdrcfhgvjb}}\label{hohwa}
界磁電流が大きくなると,エアギャップ磁束が界磁電流に比例せず,頭打ちになる現象.
同期機の磁気回路(磁束の通路)は磁極,エアギャップ,電機子歯,電機子鉄心,界磁継鉄(円筒界磁機では回転子鉄心)からなっており,このうち,エアギャップ以外はケイ素鋼板や鋼材などの強磁性体である.強磁性体の磁気分極は磁界の強さに比例せず,ある値に漸近する磁気飽和現象を有する.\\
それゆえ,同期機の磁気回路では界磁電流の増加に伴い,エアギャップの磁気抵抗は一定であるが強磁性材料部の磁気抵抗が増大するため,磁束は飽和現象を呈する.磁気飽和の程度を表すのに飽和係数を用い,無負荷飽和曲線上の定格電圧に対して,次式で表される.
\begin{equation}	
	飽和係数=\frac{鋼材部分に必要な界磁電流}{エアギャップに必要な界磁電流}
\end{equation}

\subsection{ビオ-サバールの法則(Biot-Savart's law)\cite{1130282271626280192}}
\begin{equation}
	d\boldsymbol{B}=\frac{\mu_{0}}{4\pi}\frac{Id\boldsymbol{l}\times \boldsymbol{R}}{R^{3}}
\end{equation}

\subsection{アンペールの法則(Ampère's circuital law) \cite{12}}
積分形のアンペールの法則は以下で与えられる.
\begin{equation}
	\oint_{C} \boldsymbol{\boldsymbol{r}}\cdot d\boldsymbol{l}=\mu_{0}I
\end{equation}

\subsection{電磁誘導(electromagnetic induction)\cite{titech}}
電磁誘導の法則は,時間変化する磁場中において固定した回路にもたらされる起電力がその回路を貫く磁束の時間変化に比例しているというものである.
また,磁束というものは回路に関係なく任意の曲面において定義できる.
起電力は導体の2点間の電位差として定義されるが,それは電場の線積分として表現される.任意の閉じた経路$C$を考えると 
\begin{equation}
	\oint_{C=\partial S}d\boldsymbol{r}\cdot \boldsymbol{E}(\boldsymbol{r}, t)=-\frac{d}{dt}\int_{S}d\boldsymbol{S}(\boldsymbol{r})\cdot \boldsymbol{B}(\boldsymbol{r}, t)
\end{equation}
である.$S$は$C$を境界にもつ面領域を表す.
この領域は回路に関係なく任意だが,時間に依存せず固定されているとする.
$C$が回路に一致すれば左辺は起電力を表すが,回路でなくても電場はあるので意味をもつ.
これがFaradayの法則の積分形である.

また,\weq{sto}の関係より,

\begin{equation}
	\oint_{C}d\boldsymbol{r}\cdot \boldsymbol{E}(\boldsymbol{r})=0
	\label{eq:sto}
\end{equation}
\begin{equation}
\int_{S}d\boldsymbol{S}\cdot \left(\boldsymbol{\nabla} \times \boldsymbol{E}(\boldsymbol{r}, t)+\frac{\partial}{\partial t}\boldsymbol{B}(\boldsymbol{r}, t)\right)=0
\end{equation}

よって,微分形は以下のように示される.
\begin{equation}
\boldsymbol{\nabla} \times \boldsymbol{E}(\boldsymbol{r}, t)+\frac{\partial}{\partial t}\boldsymbol{B}(\boldsymbol{r}, t)=0
\end{equation}
この公式は磁場が時間変化する場所では電場もまた時間変化しながら存在していることを意味している.
また,電場が空間的に変動しておりその回転が有限であるとき,時間依存する磁場が存在すると考えることもできる.
静電場の場合,電場は渦をつくることができず電場の回転が$0$であったが,時間変化のある場合には時間依存する磁場があるので電場の回転は$0$にならない.
時間依存性を無視すると静電場の法則に帰着する.

\subsection{抵抗損と漂遊負荷損\cite{1130282271832577152}}
ブリッジ法などで測定した巻線の抵抗(直線抵抗)を$r_{d}$とし,巻線に流れる交流電流を$I$とすると,抵抗損は$I^{2}r_{d}$である.しかし,実際の損失はこの値より大きくなる.その理由は次の2つである.
\begin{enumerate}[(1)]
	\item \textbf{導体内のうず電流損}\\
	電流による漏れ磁束が導体自身の断面に鎖交するため,導体内にうず電流が発生し,電流密度が不均一になり,導体断面積が減少したのと同じ結果となり抵抗が増加する.またその増加分は$5 \sim 20\,\%$である.また,うず電流は抵抗に反比例するため,巻線の断面積が大きい場合はうず電流を低減することができる\cite{11302822718325772}\cite{1130282270467697152}.
	\item \textbf{構造材料内の損失}\\
	漏れ磁束の一部は,タンクの側板,締め付けボルトなどを通るため,それらの部分にうず電流損失やヒステリシス損失が生じる.以上の2つを合わせて漂遊負荷損といい,抵抗損の$5 \sim 20\,\%$になる.
	抵抗損と漂遊負荷損の和が負荷損であるが,その値はほとんど電流の2乗に比例する.すなわち,二次負荷電流を$I_{2}$とすれば負荷損$W_{l}$は
	\begin{equation}
		W_{l}=I_{2}^{2}r
	\end{equation}
	で表される.
\end{enumerate}

\subsection{積分回路}
\wfig{ad}に示すようなCR回路を積分回路という.
この回路で入力信号を積分した信号を取り出すことができる.
同図で$S_{1}-S_{2}$に電源$e=E\sin \omega t$なる電圧が印加されている場合を考える.
この時回路を流れる電流$I$は$R\gg 1/\omega C$ならば\footnote{この回路の合成インピーダンスが抵抗のみであると近似できる条件}キャパシタにかかる電圧を$E_{C}$とすると
\begin{equation}
	I=\frac{E_{C}}{R}\sin \omega t
	\label{eq:I}
\end{equation}
となる.このとき$E_{C}$は,コンデンサの電荷を$q$とすれば
\begin{equation}
	E_{C}=\frac{q}{C}
	\label{eq:qcv}
\end{equation}
と表せる.また,電荷$q$は電流を積分したものであることと,\weq{I}より
\begin{align}
	q&=\int I dt\nonumber \\
	&=\int \frac{E}{R}\sin \omega tdt
	\label{eq:int}
\end{align}
\weq{qcv}と\weq{int}より
\begin{equation}
	E_{C}=\int \frac{E}{RC} \sin \omega tdt
	\label{eq:sekibun}
\end{equation}
\weq{sekibun}より,コンデンサ$C$の両端電圧$E_{C}$は入力信号$e$を積分した信号に比例した信号が得られる.

\subsection{変圧器の原理\cite{jknv}}

\subsection{変圧器の構造}

\newpage
\section{方法}
\subsection{使用器具}
今回の実験で使用した器具を\wtab{kigu}に示す.\\
なお,実験指導書ではソフトが``NI LabVIEW2015(32ビット)''と記載されていたが,使用したPCにインストールされていたものは``NI LabVIEW2019SPI(32ビット)''であった.

\begin{table}[hbtp]
  \centering
  \caption{実験装置}
  \label{tab:kigu}
  \scalebox{0.75}{
  \begin{tabular}{cccccc}
    \hline
    機器名&製造元&型番&シリアル番号&数量\,個\\
    \hline
    PC&iiyama&NK50SZ&NKNK50SZ0000K00088&1\\
    組み込みデバイス&NATIONAL INSTRUMENTS&myRIO-1900&308778E&1\\
    ブレッドボードアクセサリ&DEGILENT&MXP Breadboard for NI myRIO&D535760&1\\
    ソフトウェア&NATIONAL INSTRUMENTS&LabVIEW&LabVIEW2019 19.0.H3(32-bit)&1\\
    \hline
  \end{tabular}
  }
\end{table}

\subsection{実験手順}
プログラムの実行が速く,目視による確認が難しい場合,``ハイライトモード''を有効にすると実行過程がゆっくり表示されるようになる.
また以下では,基本的なLabVIEWの操作方法については述べない.
\subsection{実習2-1}
\subsubsection{導通確認}
\begin{enumerate}[a)]
	\item MXPブレッドボードアクセサリとジャンパーワイヤーを使って\wfig{2.12}のように回路を構築.その後,アクセサリボードをmyRIOのAポートに接続.
	\item ブロックダイアグラムで,``Analog Input''(Channel欄には,``A/AI 0(pin3)''を選択)を配置し,同右コネクタ(A/AI 0 pin3)に,``表示器''を作成.
	\item 上の計測プログラムを実行.
	\item \wfig{2.12}において``接続先を変更する''と記載されている端子をGND, 3.3\,\rm{V}, 5\,\rm{V}それぞれに接続し,値が正しく変化することを確認した.
\end{enumerate}
\begin{figure}
\centering
\includegraphics[scale=0.7]{/Users/ohyamasan/Downloads/TMCIT-Report/EE_Measurement/fig/100-volt.pdf}
\caption{定電圧計測回路}
\label{fig:2.12}
\end{figure}

\subsubsection{``For ループ''}
\begin{enumerate}[a)]
	\item 上で作成した``Analog Input''ブロックと表示器とを``Forループ''で囲むように配置し,``i''に表示器を作成.
	\item ``For ループ''のコネクタ``N''に定数を作成し5を代入.
	\item 上記プログラムを実行し正しく動作していることを確認した.
	\item ``Analog Input''の右コネクタ(A/AI 0 pin3)と``For ループ''の右端枠に接続し,コネクタに表示器を作成.(名前が``配列''に変更される)
	\item 作成したプログラムを実行し.``配列''(フロントパネル上の)に表示器(A/AI0)に表示された値が表示されることを確認した.
\end{enumerate}

\subsubsection{``配列連結追加''}
\begin{enumerate}[a)]
	\item 上の``For ループ''プログラムに``配列連結追加''ブロックを配置し,同左コネクタをカウンタ変数``i''に接続.
	\item ``配列連結追加''の左コネクタに``入力を追加''を選択し,数値入力用の左コネクタが2つになることを確認した.
	\item ``配列連結追加''ブロックの左コネクタを``Analog Input''の右コネクタ(A/AI 0 pin3)に接続.
	\item ``配列連結追加''の右コネクタを``For ループ''の右端枠に接続し,コネクタに表示器を作成.(名前が``配列2''に変更される)
	\item フロントパネルの“配列2”を選択し,5段以上表示されるように変更.
	\item 作成したプログラムを実行し,``配列2''に``i''の値と電圧値が2列で表示されることを確認した.
\end{enumerate}

\subsubsection{``配列からスプレッドシート文字列に変換''}
\begin{enumerate}[a)]
	\item ブロックダイアグラムに``配列からスプレッドシート文字列に変換''ブロックを配置.左上コネクタ(形式文字列)は空欄のまま.
	\item ブロックの左下コネクタ(配列)を``For ループ''枠右コネクタ``配列2''と接続.
	\item ブロックの被疑コネクタ(スプレッドシート文字列)に表示器を作成.(名前が``スプレッドシート文字列''に変更される)
	\item 作成したプログラムを実行し,``配列''に``i''と電圧値が2列で表示されることを確認した.
	\item フロントパネルの``スプレッドシート文字列''をトリプルクリックすると全て選択でき,Excelにペースト可能であることを確認した.
\end{enumerate}

\subsection{実習2-1 課題実験}
\begin{enumerate}[a)]
\item 100回のデータ計測をするようにプログラムを変更.
\item GND, 3.3\,\rm{V}, 5\,\rm{V}の出力電圧を100回分計測し,Excelを用いて平均値と標準偏差を求めた.
\end{enumerate}

\subsection{実習2-2}
\subsubsection{定電圧計測}
\begin{enumerate}[a)]
	\item ブロックダイアグラムに``Analog Input''を配置.Cnfiguration 画面のChannel 欄には,電圧入力ピンとして``A/AI0 (pin3)''を選択.
	\item``Analog Input''の右コネクタ(A/AI0 pin3)に``表示器''を作成.``Analog Output''を追加.
	\item Configuration 画面のChannel 欄には,電圧出力ピンとして``A/AO 0 (pin2)''を選択.
	\item ``Analog Output''の左コネクタ(A/AO 0 pin2)に``制御器''を作成.
\end{enumerate}

\subsubsection{出力電圧値変化}
\begin{enumerate}[a)]
	\item ブレッドボードアクセサリを\wfig{5-volt}のように構築し,アクセサリボードをmyRIOのAポートに挿入.
	\item 上のプログラムを実行し,``数値表示器''(A/AI0 Pin3)に表示される値を確認した.
	\item ``制御器''(A/AO 0 pin2)に0\,\rm{V}から5\,\rm{V}の数値を入力する.複数回実行し,``数値表示器''(A/AI0 pin3)に表示される値を確認した.
	\item ``制御器''(A/AO 0 pin2)の数値をさらに変更し,上と同様に複数回実行し,値の変化を確認した.
\end{enumerate}

\begin{figure}[htb]
\centering
\includegraphics[scale=0.65]{/Users/ohyamasan/Downloads/TMCIT-Report/EE_Measurement/fig/5-volt.pdf}
\caption{変電圧計測回路}
\label{fig:5-volt}
\end{figure}

\subsubsection{出力電圧表示値と計測電圧値を配列・文字列で表示}
\begin{enumerate}[a)]
	\item 上記のプログラムの全てのプログラムを``For ループ''で囲むように配置.コネクタ``N''に定数を作成し,5を代入.
	\item ``Analog Output''の左コネクタ(A/AO 0 pin2)をカウンタ変数``i''に接続.(制御器``AO0''との接続は予め解除しておく)
	\item ``配列連結追加''ブロックを追加し,その出力コネクタから伸ばしたワイヤーを``For ループ枠''に繋げ,表示器を新規作成.
	\item 作成したプログラムを実行し,``配列''に表示される電圧値を予想される値と比較した.
	\item ブロックダイアグラムに``遅延時間''を追加し,制御器を追加.``遅延時間''は0.05\,秒に設定.
	\item ``AnalogOutput''の``error out''と``遅延時間''の``エラー入力'',``遅延時間''の``エラー出力''と``AnalogOutput''の``error in''を接続.
	\item ``遅延時間''ブロックの追加による``配列''に表示される値の変化を確認した.
	\item 遅延時間を,0\,秒, 0.005\,秒, 0.050\,秒, 0.500\,秒に変え,それぞれの場合で入力値を変更し,表示値との比較を行った.(3.5.2のように2回目以降は入力値と出力値がほぼ一致した)
	\item 以降の手順は3.3.3および3.3.4を参照し,文字列を表示させる.
\end{enumerate}


\subsection{実習2-2 課題実験}
\begin{enumerate}[a)]
	\item 0\,\rm{V}から5\,\rm{V}まで0.5\,\rm{V}刻みで出力電圧を変え,出力電圧値を計測.
	\item ``Analog Output''に入力した値・``Analog Input''から取得した値をカウンタ変数``i'', ``Analog Output''表示値・``Analog Input''表示値を表示.
	\item ``出力電圧表示値''と``計測電圧表示値''との差を表示.
	\item 上で導出した差から二乗平均平方根誤差(Root Mean Squared Error, RMSE)をExcelを用いて求めた.
\end{enumerate}


\subsection{実習3}
\subsubsection{固定抵抗の電圧電流特性}
\begin{enumerate}[a)]
	\item \wfig{mesure-vl}のように回路を構築.
	\item V, $V_{U}$, $R_{0}$を用いて電圧,電流を計測し電圧-電流特性を計測する.なお測定対象素子は,1\,k\rm{$\Omega$}.$R_{0}$の抵抗値変化は,100\,\rm{$\Omega$},1\,k\rm{$\Omega$},10\,k\rm{$\Omega$},100\,k\rm{$\Omega$}である.
\end{enumerate}

\begin{figure}[h]
\centering
\includegraphics[scale=0.75]{/Users/ohyamasan/Downloads/TMCIT-Report/EE_Measurement/fig/mesure-vl.pdf}
\caption{電圧電流特性計測回路}
\label{fig:mesure-vl}
\end{figure}


\subsubsection{可変抵抗(ポテンショメータ)の電圧電流特性}
\begin{enumerate}[a)]
	\item \wfig{mesure-vl}において,測定素子を可変抵抗に変更.
	\item \wfig{poten}を参考にして,可変抵抗の端子間を1-2としてつまみ位置をA, B, C (くぼみが向いている向きがA, B, Cとなるように.すなわち,図ではAの状態である)にして,それぞれの電圧を計測した.なお,プログラムは上で作成したものをそのまま利用した.
	\item 可変抵抗の端子間を2-3, 3-1間でも同様に測定した.
	\item ただし,$R_{0}$は測定データを基に適切なものを選択した.
\end{enumerate}

\begin{figure}[h]
\centering
\includegraphics[scale=0.6]{/Users/ohyamasan/Downloads/TMCIT-Report/EE_Measurement/fig/potencial.pdf}
\caption{ポテンショメータ}
\label{fig:poten}
\end{figure}

\subsubsection{CdSセンサの電圧電流特性}
\begin{enumerate}[a)]
	\item \wfig{mesure-vl}の測定素子をCdSセンサ(\wfig{CdS})に変更.
	\item 自然状態とスマホのライトをセンサ部分に照射する場合のそれぞれで,上と同様に測定を行った.
	\item ただし,$R_{0}$は測定データを基に適切なものを選択した.
\end{enumerate}

\begin{figure}[h]
\centering
\includegraphics[scale=0.5]{/Users/ohyamasan/Downloads/TMCIT-Report/EE_Measurement/fig/31AbZ9saczL._AC_.jpg}
\caption{CdSセンサ\cite{cbs}}
\label{fig:CdS}
\end{figure}

\subsubsection{力センサの電圧電流特性}
\begin{enumerate}[a)]
	\item  \wfig{mesure-vl}の測定素子を力センサ(\wfig{power})に変更.
	\item 自然状態とセンサ部分(黒い枠で囲まれている円部分)を強く押し続けた状態のそれぞれで,上と同様に測定を行った.
	\item ただし,$R_{0}$は測定データを基に適切なものを選択した.
\end{enumerate}

\begin{figure}[h]
\centering
\includegraphics[scale=0.5]{/Users/ohyamasan/Downloads/TMCIT-Report/EE_Measurement/fig/31Gu87ucreL.jpg}
\caption{力センサ\cite{power}}
\label{fig:power}
\end{figure}

\subsubsection{発光ダイオードの電圧電流特性}
\begin{enumerate}[a)]
	\item \wfig{mesure-vl}の測定素子をLEDに変更.
	\item 測定するLEDは発光色4種類(赤,青,白,緑)で行った.
	\item ただし,$R_{0}$は測定データを基に適切なものを選択した.
\end{enumerate}
\newpage
\section{結果}
\subsection{実習2-1 課題実験}
\begin{itemize}
\item プログラム\ref{ex21-block}は定電圧を100回測定するプログラムである.このようにNの数を99にすることで100回のデータ測定が可能となる.
\item 実行後,フロントパネルでは,\wfig{ex21-flont}のように表示された.
\item \wtab{GND}から\wtab{fiveV}にGND,3.3\,\rm{V},5\,\rm{V}それぞれに接続した際の出力電圧を示す.
また,\wtab{means}は計測データから算出された平均値・標準偏差である.算出方法は\weq{mean},\weq{nut},\weq{keisan-nut}の通りである\cite{7d69cf97-0c1f}.導出過程で利用したExcel関数\weq{SUM}, \weq{SQRT}はそれぞれ,$\sum_{A}^{B}$, $\sqrt{A}$に対応している.
\begin{align}
平均値\,\rm{V}&:=(SUM(X_{0}:X_{99}))/100 (=MEAN_{v})\label{eq:mean}\\
標準偏差 - 推定値\,\rm{V}&:=STDEV.S(X_{0}:X_{99})\label{eq:nut}\\
標準偏差 - 計算値\,\rm{V}&:=SQRT((X_{i}-MEAN_{v})^2/99)\label{eq:keisan-nut}\\
X_{i}&:データのi番目 \nonumber \\
MEAN_{v}&:上で計算した平均値をとる定数.GND, 3.3\,\rm{V}, 5\,\rm{V}の場合の計3種類\nonumber\\
SUM(A:B)&:AからBまでの和を返すExcel関数\label{eq:SUM}\\
STDEV.S(A:B)&:引数AからBをもとに,母集団の標準偏差の推定値を返すExcel関数\label{eq:STDEV.S}\\
SQRT(A)&:Aの平方根を返すExcel関数\label{eq:SQRT}
\end{align}
\end{itemize}

\begin{figure}[h]
\centering
\includegraphics[scale=0.5]{./fig/ex21-block.pdf}\\
\useMycounter[\label{ex21-block}]定電圧計測のブロックダイアグラム
\end{figure}

\begin{table}[h]
\centering
\caption{Excelを用いて算出された平均値と標準偏差}
\label{tab:means}
\begin{tabular}{cccc}
\hline
    接続先 	\textbackslash 算出値 &平均値[\rm{V}] &標準偏差 - 推定値[\rm{V}]   & 標準偏差 - 計算値[\rm{V}]    \\
     \hline
  GND& 0.00636  & 0.000499  & 0.000499419 \\
3.3\,\rm{V} & 3.26256 & 0.000568 & 0.000567549 \\
 5\,\rm{V}& 4.998779 & 0.00000000000000981918 &  0.00000000000000981918 \\
\hline
\end{tabular}
\end{table}

\begin{figure}[h]
\centering
\includegraphics[scale=0.35]{./fig/ex21-flont.pdf}
\caption{定電圧計測のフロントパネル}
\label{fig:ex21-flont}
\end{figure}

\begin{table}[t]
\centering
 \caption{GND接続時の出力電圧}
 \label{tab:GND}
 \scalebox{0.7}{
\begin{tabular}{cccccccc}
\hline
測定回数[\rm{回}]&出力電圧[\rm{V}]&測定回数[\rm{回}]&出力電圧[\rm{V}]&測定回数[\rm{回}] &出力電圧[\rm{V}] &測定回数[\rm{回}] &出力電圧[\rm{V}] \\
\hline
0  & 0.006104 & 25 & 0.006104 & 50 & 0.006104 & 75 & 0.006104 \\
1  & 0.007324 & 26 & 0.007324 & 51 & 0.006104 & 76 & 0.006104 \\
2  & 0.006104 & 27 & 0.007324 & 52 & 0.006104 & 77 & 0.006104 \\
3  & 0.007324 & 28 & 0.007324 & 53 & 0.006104 & 78 & 0.006104 \\
4  & 0.007324 & 29 & 0.007324 & 54 & 0.006104 & 79 & 0.006104 \\
5  & 0.006104 & 30 & 0.007324 & 55 & 0.006104 & 80 & 0.006104 \\
6  & 0.006104 & 31 & 0.006104 & 56 & 0.006104 & 81 & 0.006104 \\
7  & 0.006104 & 32 & 0.006104 & 57 & 0.006104 & 82 & 0.007324 \\
8  & 0.006104 & 33 & 0.006104 & 58 & 0.006104 & 83 & 0.007324 \\
9  & 0.006104 & 34 & 0.006104 & 59 & 0.006104 & 84 & 0.006104 \\
10 & 0.006104 & 35 & 0.007324 & 60 & 0.006104 & 85 & 0.006104 \\
11 & 0.006104 & 36 & 0.007324 & 61 & 0.006104 & 86 & 0.007324 \\
12 & 0.006104 & 37 & 0.006104 & 62 & 0.006104 & 87 & 0.007324 \\
13 & 0.007324 & 38 & 0.006104 & 63 & 0.006104 & 88 & 0.006104 \\
14 & 0.007324 & 39 & 0.006104 & 64 & 0.006104 & 89 & 0.006104 \\
15 & 0.007324 & 40 & 0.006104 & 65 & 0.006104 & 90 & 0.006104 \\
16 & 0.006104 & 41 & 0.006104 & 66 & 0.006104 & 91 & 0.006104 \\
17 & 0.006104 & 42 & 0.006104 & 67 & 0.006104 & 92 & 0.006104 \\
18 & 0.006104 & 43 & 0.006104 & 68 & 0.006104 & 93 & 0.006104 \\
19 & 0.006104 & 44 & 0.006104 & 69 & 0.006104 & 94 & 0.006104 \\
20 & 0.006104 & 45 & 0.006104 & 70 & 0.006104 & 95 & 0.006104 \\
21 & 0.006104 & 46 & 0.006104 & 71 & 0.006104 & 96 & 0.006104 \\
22 & 0.007324 & 47 & 0.006104 & 72 & 0.006104 & 97 & 0.007324 \\
23 & 0.007324 & 48 & 0.006104 & 73 & 0.006104 & 98 & 0.007324 \\
24 & 0.006104 & 49 & 0.006104 & 74 & 0.006104 & 99 & 0.006104\\
\hline
\end{tabular}
}
 \end{table}
 
\begin{table}[b]
 \centering
 \caption{3.3\,\rm{V}接続時の出力電圧}
 \label{tab:three-threeV}
\scalebox{0.7}{
\begin{tabular}{cccccccc}
\hline
測定回数[\rm{回}]&出力電圧[\rm{V}]&測定回数[\rm{回}]&出力電圧[\rm{V}]&測定回数[\rm{回}] &出力電圧[\rm{V}] &測定回数[\rm{回}] &出力電圧[\rm{V}] \\
\hline
0  & 3.262939 & 25 & 3.262939 & 50 & 3.261718 & 75 & 3.262939 \\
1  & 3.262939 & 26 & 3.262939 & 51 & 3.262939 & 76 & 3.262939 \\
2  & 3.262939 & 27 & 3.262939 & 52 & 3.262939 & 77 & 3.262939 \\
3  & 3.262939 & 28 & 3.262939 & 53 & 3.262939 & 78 & 3.262939 \\
4  & 3.262939 & 29 & 3.262939 & 54 & 3.262939 & 79 & 3.261718 \\
5  & 3.261718 & 30 & 3.262939 & 55 & 3.262939 & 80 & 3.261718 \\
6  & 3.261718 & 31 & 3.262939 & 56 & 3.262939 & 81 & 3.262939 \\
7  & 3.262939 & 32 & 3.262939 & 57 & 3.262939 & 82 & 3.262939 \\
8  & 3.262939 & 33 & 3.262939 & 58 & 3.262939 & 83 & 3.262939 \\
9  & 3.262939 & 34 & 3.261718 & 59 & 3.262939 & 84 & 3.262939 \\
10 & 3.262939 & 35 & 3.261718 & 60 & 3.262939 & 85 & 3.261718 \\
11 & 3.262939 & 36 & 3.261718 & 61 & 3.262939 & 86 & 3.261718 \\
12 & 3.262939 & 37 & 3.261718 & 62 & 3.262939 & 87 & 3.262939 \\
13 & 3.261718 & 38 & 3.261718 & 63 & 3.262939 & 88 & 3.262939 \\
14 & 3.261718 & 39 & 3.261718 & 64 & 3.262939 & 89 & 3.262939 \\
15 & 3.261718 & 40 & 3.261718 & 65 & 3.262939 & 90 & 3.262939 \\
16 & 3.261718 & 41 & 3.262939 & 66 & 3.262939 & 91 & 3.262939 \\
17 & 3.262939 & 42 & 3.262939 & 67 & 3.262939 & 92 & 3.262939 \\
18 & 3.262939 & 43 & 3.261718 & 68 & 3.262939 & 93 & 3.262939 \\
19 & 3.262939 & 44 & 3.261718 & 69 & 3.262939 & 94 & 3.262939 \\
20 & 3.262939 & 45 & 3.261718 & 70 & 3.262939 & 95 & 3.262939 \\
21 & 3.262939 & 46 & 3.261718 & 71 & 3.261718 & 96 & 3.262939 \\
22 & 3.261718 & 47 & 3.262939 & 72 & 3.261718 & 97 & 3.262939 \\
23 & 3.261718 & 48 & 3.262939 & 73 & 3.261718 & 98 & 3.261718 \\
24 & 3.262939 & 49 & 3.261718 & 74 & 3.261718 & 99 & 3.261718 \\
\hline
\end{tabular}
}
\end{table}

\begin{table}[t]
\centering
 \caption{5\,\rm{V}接続時の出力電圧}
 \label{tab:fiveV}
 \scalebox{0.7}{
\begin{tabular}{cccccccc}
\hline
測定回数[\rm{回}]&出力電圧[\rm{V}]&測定回数[\rm{回}]&出力電圧[\rm{V}]&測定回数[\rm{回}]&出力電圧[\rm{V}]&測定回数[\rm{回}]&出力電圧[\rm{V}]\\
\hline
0  & 4.998779 & 25 & 4.998779 & 50 & 4.998779 & 75 & 4.998779 \\
1  & 4.998779 & 26 & 4.998779 & 51 & 4.998779 & 76 & 4.998779 \\
2  & 4.998779 & 27 & 4.998779 & 52 & 4.998779 & 77 & 4.998779 \\
3  & 4.998779 & 28 & 4.998779 & 53 & 4.998779 & 78 & 4.998779 \\
4  & 4.998779 & 29 & 4.998779 & 54 & 4.998779 & 79 & 4.998779 \\
5  & 4.998779 & 30 & 4.998779 & 55 & 4.998779 & 80 & 4.998779 \\
6  & 4.998779 & 31 & 4.998779 & 56 & 4.998779 & 81 & 4.998779 \\
7  & 4.998779 & 32 & 4.998779 & 57 & 4.998779 & 82 & 4.998779 \\
8  & 4.998779 & 33 & 4.998779 & 58 & 4.998779 & 83 & 4.998779 \\
9  & 4.998779 & 34 & 4.998779 & 59 & 4.998779 & 84 & 4.998779 \\
10 & 4.998779 & 35 & 4.998779 & 60 & 4.998779 & 85 & 4.998779 \\
11 & 4.998779 & 36 & 4.998779 & 61 & 4.998779 & 86 & 4.998779 \\
12 & 4.998779 & 37 & 4.998779 & 62 & 4.998779 & 87 & 4.998779 \\
13 & 4.998779 & 38 & 4.998779 & 63 & 4.998779 & 88 & 4.998779 \\
14 & 4.998779 & 39 & 4.998779 & 64 & 4.998779 & 89 & 4.998779 \\
15 & 4.998779 & 40 & 4.998779 & 65 & 4.998779 & 90 & 4.998779 \\
16 & 4.998779 & 41 & 4.998779 & 66 & 4.998779 & 91 & 4.998779 \\
17 & 4.998779 & 42 & 4.998779 & 67 & 4.998779 & 92 & 4.998779 \\
18 & 4.998779 & 43 & 4.998779 & 68 & 4.998779 & 93 & 4.998779 \\
19 & 4.998779 & 44 & 4.998779 & 69 & 4.998779 & 94 & 4.998779 \\
20 & 4.998779 & 45 & 4.998779 & 70 & 4.998779 & 95 & 4.998779 \\
21 & 4.998779 & 46 & 4.998779 & 71 & 4.998779 & 96 & 4.998779 \\
22 & 4.998779 & 47 & 4.998779 & 72 & 4.998779 & 97 & 4.998779 \\
23 & 4.998779 & 48 & 4.998779 & 73 & 4.998779 & 98 & 4.998779 \\
24 & 4.998779 & 49 & 4.998779 & 74 & 4.998779 & 99 & 4.998779\\
\hline
\end{tabular}
}
\end{table}

\clearpage
\subsection{実習2-2 課題実験}
\begin{itemize}
	\item プログラム\ref{ex22-block}は変電圧測定プログラムである.これのようにNの値を11に,カウンタ変数を2で除することで0\,\rm{V}から5\,\rm{V}まで0.5\,\rm{V}刻みで出力電圧を計測することが可能となる.
	\item 実行した後のフロントパネルを\wfig{ex22-flont}に示す.また,\weq{soutaigosa}に基づいて計算した相対誤差を\wtab{fivegosa}に示す.
表からもわかるように0\,\rm{V}の際の相対誤差が飛び抜けて大きいことが読み取れる.	
	\item Excel を用いて計算したRMSEを\wtab{RMSE}に示す\cite{lm-evaluation}.
	\item RMSEの計算式は\weq{lm-evaluation},Excelでは\weq{eRMSE}で計算した.
\end{itemize}

\begin{align}
RMSE&=\sqrt{\frac{1}{n}\sum_{i=1}^{n}(y_{i}-\hat{y_{i}})^{2}}\label{eq:lm-evaluation}\\
y_{i}:観測値(n個)\quad&\qquad\hat{y_{i}}:計算値(n個)\nonumber\\
RMSE&=SQRT(1/11)*SQRT(SUM(\overline{X_{i}}^{2}))\label{eq:eRMSE}\\
\overline{X_{i}}&:X_{i}での差(誤差)\nonumber
\end{align}

\begin{figure}[h]
\centering
\includegraphics[scale=0.5]{./fig/ex22-block.pdf}\\
\useMycounter[\label{ex22-block}]出力電圧変化測定時のブロックダイアグラム
\end{figure}
\begin{figure}[h]
\centering
\includegraphics[scale=0.5]{./fig/ex22-flont.pdf}
\caption{出力電圧変化測定時のフロントパネル}
\label{fig:ex22-flont}
\end{figure}

\begin{table}[h]
\begin{minipage}[c]{0.5\hsize}
\centering
\caption{電圧変化測定時の相対誤差}
\label{tab:fivegosa}
\begin{tabular}{cc}
\hline
電圧[\rm{V}]  & 相対誤差[\rm{V}]\\
\hline
0   & 0  \\
0.5 & 0  \\
1   & -1 \\
1.5 & -1 \\
2   & -2 \\
2.5 & -3 \\
3   & -3 \\
3.5 & -3 \\
4   & -4 \\
4.5 & -4 \\
5   & -5 \\
\hline
\end{tabular}
\end{minipage}
\begin{minipage}[c]{0.5\hsize}
\centering
\caption{電圧変化測定時のRMSE}
\label{tab:RMSE}
\begin{tabular}{cc}
\hline
電圧[\rm{V}]  & 二乗平均平方根誤差[\rm{V}]  \\
\hline
0   & 0.002576414 \\
0.5 & 0.000956997 \\
1   & 0.000809859 \\
1.5 & 0.001030566 \\
2   & 0.000515283 \\
2.5 & 0.000367844 \\
3   & 0.000221008 \\
3.5 & 0.0000735688 \\
4   & 0.000441413 \\
4.5 & 0.000147439 \\
5   & 0.000368145\\
\hline
\end{tabular}
\end{minipage}
\end{table}
\clearpage
\subsection{実習3-1 固定抵抗の電圧電流特性}
\begin{itemize}
	\item プログラム\ref{ex31-block}は$R_{0}$の場合の,固定抵抗の電圧電流特性(0-5\,\rm{V}まで0.1\,\rm{V}刻み)を計測するプログラムである.
	\item 上図の中心部分の100という数字は$R_{0}$の値にするとよい.
	\item 実行後,フロントパネルは\wfig{ex31-flont}のようになり,計測結果は\wtab{R}のようになった.
	\item 計測データを基に作成した電圧電流特性のグラフは\wfig{3-1}である.
\end{itemize}

\begin{figure}[h]
\centering
\includegraphics[scale=0.5]{./fig/ex31-block.pdf}\\
\useMycounter[\label{ex31-block}]固定抵抗の電圧電流特性計測時のブロックダイアグラム
\end{figure}
\begin{figure}[h]
\centering
\includegraphics[scale=0.5]{./fig/ex31-flont.pdf}
\caption{固定抵抗の電圧電流特性測定時のフロントパネル}
\label{fig:ex31-flont}
\end{figure}

\begin{table}[h]
    \centering
     \caption{固定抵抗の諸特性値}    
     \label{tab:R}
     \scalebox{0.8}{
    \begin{tabular}{cccccc}
\hline
カウンタ変数 & 出力電圧[\rm{V}] & 全電圧[\rm{V}]  & $V_{R0}$[\rm{V}]   & $V_{U}$[\rm{V}]    & $I_{U}$[\rm{A}]  \\
\hline
  0      & 0    & 0.007324 & 0        & 0.007324 & 0        \\
  1      & 0.1  & 0.096436 & 0.007324 & 0.089111 & 0.0000732 \\
  2      & 0.2  & 0.197754 & 0.019531 & 0.178223 & 0.000195 \\
  3      & 0.3  & 0.297852 & 0.026855 & 0.270996 & 0.000269 \\
  4      & 0.4  & 0.39917  & 0.037842 & 0.361328 & 0.000378 \\
  5      & 0.5  & 0.498047 & 0.046387 & 0.45166  & 0.000464 \\
  6      & 0.6  & 0.595703 & 0.05249  & 0.543213 & 0.000525 \\
  7      & 0.7  & 0.697021 & 0.063477 & 0.633545 & 0.000635 \\
  8      & 0.8  & 0.797119 & 0.073242 & 0.723877 & 0.000732 \\
  9      & 0.9  & 0.897217 & 0.084229 & 0.812988 & 0.000842 \\
  10     & 1    & 0.996094 & 0.091553 & 0.904541 & 0.000916 \\
  11     & 1.1  & 1.096191 & 0.101318 & 0.994873 & 0.001013 \\
  12     & 1.2  & 1.195068 & 0.108643 & 1.086426 & 0.001086 \\
  13     & 1.3  & 1.297607 & 0.12207  & 1.175537 & 0.001221 \\
  14     & 1.4  & 1.396484 & 0.128174 & 1.26831  & 0.001282 \\
  15     & 1.5  & 1.496582 & 0.13916  & 1.357422 & 0.001392 \\
  16     & 1.6  & 1.5979   & 0.150146 & 1.447754 & 0.001501 \\
  17     & 1.7  & 1.696777 & 0.157471 & 1.539306 & 0.001575 \\
  18     & 1.8  & 1.795654 & 0.164795 & 1.630859 & 0.001648 \\
  19     & 1.9  & 1.895752 & 0.174561 & 1.721191 & 0.001746 \\
  20     & 2    & 1.994629 & 0.183105 & 1.811523 & 0.001831 \\
  21     & 2.1  & 2.094726 & 0.192871 & 1.901855 & 0.001929 \\
  22     & 2.2  & 2.194824 & 0.202637 & 1.992187 & 0.002026 \\
  23     & 2.3  & 2.294922 & 0.211182 & 2.08374  & 0.002112 \\
  24     & 2.4  & 2.395019 & 0.220947 & 2.174072 & 0.002209 \\
  25     & 2.5  & 2.495117 & 0.230713 & 2.264404 & 0.002307 \\
  26     & 2.6  & 2.593994 & 0.238037 & 2.355957 & 0.00238  \\
  27     & 2.7  & 2.694092 & 0.247803 & 2.446289 & 0.002478 \\
  28     & 2.8  & 2.794189 & 0.257568 & 2.536621 & 0.002576 \\
  29     & 2.9  & 2.894287 & 0.266113 & 2.628174 & 0.002661 \\
  30     & 3    & 2.995605 & 0.2771   & 2.718506 & 0.002771 \\
  31     & 3.1  & 3.094482 & 0.285645 & 2.808838 & 0.002856 \\
  32     & 3.2  & 3.193359 & 0.29541  & 2.897949 & 0.002954 \\
  33     & 3.3  & 3.293457 & 0.302734 & 2.990722 & 0.003027 \\
  34     & 3.4  & 3.393554 & 0.3125   & 3.081054 & 0.003125 \\
  35     & 3.5  & 3.492431 & 0.321045 & 3.171386 & 0.00321  \\
  36     & 3.6  & 3.59497  & 0.333252 & 3.261718 & 0.003333 \\
  37     & 3.7  & 3.693847 & 0.339355 & 3.354492 & 0.003394 \\
  38     & 3.8  & 3.793945 & 0.349121 & 3.444824 & 0.003491 \\
  39     & 3.9  & 3.894043 & 0.358887 & 3.535156 & 0.003589 \\
  40     & 4    & 3.99292  & 0.367432 & 3.625488 & 0.003674 \\
  41     & 4.1  & 4.091796 & 0.374756 & 3.717041 & 0.003748 \\
  42     & 4.2  & 4.193115 & 0.385742 & 3.807373 & 0.003857 \\
  43     & 4.3  & 4.274902 & 0.390625 & 3.884277 & 0.003906 \\
  44     & 4.4  & 4.272461 & 0.391846 & 3.880615 & 0.003918 \\
  45     & 4.5  & 4.270019 & 0.390625 & 3.879394 & 0.003906 \\
  46     & 4.6  & 4.268798 & 0.390625 & 3.878173 & 0.003906 \\
  47     & 4.7  & 4.266357 & 0.390625 & 3.875732 & 0.003906 \\
  48     & 4.8  & 4.266357 & 0.391846 & 3.874511 & 0.003918 \\
  49     & 4.9  & 4.263916 & 0.390625 & 3.873291 & 0.003906 \\
  50     & 5    & 4.262695 & 0.389404 & 3.873291 & 0.003894 \\
\hline
\end{tabular}
}
\end{table}

\begin{figure}[h]
        \centering
        \includegraphics[scale=0.45]{./fig/3-1.pdf}
	\caption{固定抵抗の電圧電流特性}
	\label{fig:3-1}  
\end{figure}
\clearpage
\subsection{実習3-2 可変抵抗(ポテンショメータ)の電圧電流特性}
\begin{itemize}
	\item プログラム\ref{ex32-block}は可変抵抗端子1-2間の電圧電流特性を計測するプログラムである.
	\item 全ての場合において,$R_{0}$の値は10\,k\rm{$\Omega$}とした.
	\item 実行後,フロントパネルは\wfig{ex32-flont}のようになり,計測結果は\wtab{ER1-2}のようになった.
	\item 計測データを基に作成した電圧電流特性のグラフは\wfig{3-2-1}である.
	\item 端子2-3間では,\wtab{ER2-3},\wfig{3-2-2}.端子3-1間では,\wtab{ER3-1},\wfig{3-2-3}となった.
\end{itemize}

\begin{figure}[h]
\centering
\includegraphics[scale=0.5]{./fig/ex32-block.pdf}\\
\useMycounter[\label{ex32-block}]可変抵抗の電圧電流特性計測時のブロックダイアグラム
\end{figure}

\begin{figure}[h]
\centering
\includegraphics[scale=0.5]{./fig/ex32-flont.pdf}
\caption{可変抵抗の電圧電流特性測定時のフロントパネル}
\label{fig:ex32-flont}
\end{figure}

\begin{table}[h]
\centering
\caption{可変抵抗器(1-2端子間)の諸特性値}
\label{tab:ER1-2}
\scalebox{0.8}{
\begin{tabular}{ccccccc}
\hline
カウンタ変数 & Aでの$V_{U}$[\rm{V}] & Aでの$I_{U}$[\rm{A}] & Bでの$V_{U}$[\rm{V}]    & Bでの$I_{U}$[\rm{A}]   & Cでの$V_{U}$[\rm{V}]& Cでの$I_{U}$[\rm{A}] \\
\hline
  0      & 0.007324 & 0.00000012 & 0.006104 & 0.00000024 & 0.006104 & 0.00000024 \\
  1      & 0.048828 & 0.00000513 & 0.031738 & 0.00000696 & 0.007324 & 0.00000903 \\
  2      & 0.098877 & 0.00000977 & 0.065918 & 0.00001306 & 0.006104 & 0.00001892 \\
  3      & 0.151367 & 0.00001453 & 0.100098 & 0.00001953 & 0.006104 & 0.00002930 \\
  4      & 0.203857 & 0.00001929 & 0.133057 & 0.00002637 & 0.006104 & 0.00003894 \\
  5      & 0.253906 & 0.00002441 & 0.168457 & 0.00003296 & 0.006104 & 0.00004932 \\
  6      & 0.305176 & 0.00002930 & 0.201416 & 0.00003979 & 0.007324 & 0.00005920 \\
  7      & 0.355225 & 0.00003418 & 0.234375 & 0.00004639 & 0.007324 & 0.00006897 \\
  8      & 0.406494 & 0.00003918 & 0.268555 & 0.00005286 & 0.007324 & 0.00007910 \\
  9      & 0.456543 & 0.00004407 & 0.301514 & 0.00005957 & 0.007324 & 0.00008911 \\
  10     & 0.507812 & 0.00004907 & 0.336914 & 0.00006616 & 0.007324 & 0.00009888 \\
  11     & 0.559082 & 0.00005396 & 0.369873 & 0.00007275 & 0.006104 & 0.00010900 \\
  12     & 0.610352 & 0.00005859 & 0.404053 & 0.00007947 & 0.006104 & 0.00011900 \\
  13     & 0.661621 & 0.00006360 & 0.438232 & 0.00008606 & 0.006104 & 0.00012900 \\
  14     & 0.712891 & 0.00006848 & 0.472412 & 0.00009253 & 0.006104 & 0.00013900 \\
  15     & 0.762939 & 0.00007349 & 0.505371 & 0.00009924 & 0.006104 & 0.00014900 \\
  16     & 0.814209 & 0.00007849 & 0.539551 & 0.00010600 & 0.006104 & 0.00015900 \\
  17     & 0.865478 & 0.00008350 & 0.57251  & 0.00011300 & 0.007324 & 0.00016900 \\
  18     & 0.915527 & 0.00008838 & 0.606689 & 0.00011900 & 0.007324 & 0.00017900 \\
  19     & 0.966797 & 0.00009302 & 0.640869 & 0.00012600 & 0.007324 & 0.00018900 \\
  20     & 1.018066 & 0.00009790 & 0.673828 & 0.00013200 & 0.007324 & 0.00019900 \\
  21     & 1.068115 & 0.00010300 & 0.709228 & 0.00013900 & 0.007324 & 0.00020900 \\
  22     & 1.119385 & 0.00010800 & 0.742187 & 0.00014600 & 0.008545 & 0.00021900 \\
  23     & 1.170654 & 0.00011300 & 0.775146 & 0.00015200 & 0.008545 & 0.00022900 \\
  24     & 1.221924 & 0.00011800 & 0.809326 & 0.00015900 & 0.008545 & 0.00023900 \\
  25     & 1.273193 & 0.00012200 & 0.843506 & 0.00016600 & 0.008545 & 0.00024900 \\
  26     & 1.324463 & 0.00012700 & 0.877685 & 0.00017200 & 0.008545 & 0.00025900 \\
  27     & 1.374512 & 0.00013200 & 0.910644 & 0.00017900 & 0.009766 & 0.00026900 \\
  28     & 1.425781 & 0.00013700 & 0.944824 & 0.00018600 & 0.009766 & 0.00027900 \\
  29     & 1.477051 & 0.00014200 & 0.979004 & 0.00019200 & 0.008545 & 0.00028900 \\
  30     & 1.527099 & 0.00014700 & 1.013183 & 0.00019900 & 0.009766 & 0.00029900 \\
  31     & 1.57959  & 0.00015200 & 1.046142 & 0.00020500 & 0.010986 & 0.00030900 \\
  32     & 1.630859 & 0.00015700 & 1.080322 & 0.00021200 & 0.012207 & 0.00031900 \\
  33     & 1.680908 & 0.00016200 & 1.114502 & 0.00021900 & 0.010986 & 0.00032900 \\
  34     & 1.730957 & 0.00016700 & 1.147461 & 0.00022500 & 0.013428 & 0.00033900 \\
  35     & 1.782226 & 0.00017200 & 1.181641 & 0.00023200 & 0.012207 & 0.00034900 \\
  36     & 1.834717 & 0.00017600 & 1.21582  & 0.00023800 & 0.013428 & 0.00035900 \\
  37     & 1.885986 & 0.00018100 & 1.25     & 0.00024500 & 0.013428 & 0.00036900 \\
  38     & 1.936035 & 0.00018600 & 1.28418  & 0.00025200 & 0.014648 & 0.00037900 \\
  39     & 1.987304 & 0.00019100 & 1.317139 & 0.00025800 & 0.014648 & 0.00038900 \\
  40     & 2.038574 & 0.00019600 & 1.351318 & 0.00026500 & 0.014648 & 0.00039800 \\
  41     & 2.089844 & 0.00020100 & 1.385498 & 0.00027200 & 0.015869 & 0.00040800 \\
  42     & 2.141113 & 0.00020600 & 1.419678 & 0.00027800 & 0.015869 & 0.00041800 \\
  43     & 2.192383 & 0.00021100 & 1.453857 & 0.00028500 & 0.015869 & 0.00042800 \\
  44     & 2.241211 & 0.00021600 & 1.486816 & 0.00029100 & 0.015869 & 0.00043800 \\
  45     & 2.29248  & 0.00022100 & 1.520996 & 0.00029800 & 0.015869 & 0.00044800 \\
  46     & 2.34497  & 0.00022500 & 1.555176 & 0.00030400 & 0.01709  & 0.00045800 \\
  47     & 2.39624  & 0.00023000 & 1.589355 & 0.00031100 & 0.01709  & 0.00046800 \\
  48     & 2.446289 & 0.00023500 & 1.623535 & 0.00031800 & 0.01709  & 0.00047800 \\
49     & 2.495117 & 0.00024100 & 1.657715 & 0.00032400 & 0.01709  & 0.00048800 \\
50     & 2.547607 & 0.00024500 & 1.689453 & 0.00033100 & 0.018311 & 0.00049800 \\
\hline
\end{tabular}
}
\end{table}

\begin{table}[h]
\centering
\caption{可変抵抗器(2-3端子間)の諸特性値}
\label{tab:ER2-3}
     \scalebox{0.8}{
    \begin{tabular}{ccccccc}
    \hline
	カウンタ変数 & Aでの$V_{U}$[\rm{V}] & Aでの$I_{U}$[\rm{A}] & Bでの$V_{U}$[\rm{V}]   & Bでの$I_{U}$[\rm{A}]   & Cでの$V_{U}$[\rm{V}] & Cでの$I_{U}$[\rm{A}]\\
	\hline
	0      & 0.006104 & 0.00000024 & 0.006104 & 0.00000012 & 0.007324 & 0.00000000 \\
	1      & 0.007324 & 0.00000891 & 0.0354   & 0.00000610 & 0.050049 & 0.00000488 \\
	2      & 0.007324 & 0.00001892 & 0.070801 & 0.00001257 & 0.101318 & 0.00000964 \\
	3      & 0.009766 & 0.00002881 & 0.108643 & 0.00001868 & 0.151367 & 0.00001453 \\
	4      & 0.012207 & 0.00003870 & 0.144043 & 0.00002539 & 0.202637 & 0.00001953 \\
	5      & 0.015869 & 0.00004822 & 0.180664 & 0.00003174 & 0.253906 & 0.00002441 \\
	6      & 0.019531 & 0.00005774 & 0.217285 & 0.00003796 & 0.305176 & 0.00002930 \\
	7      & 0.023193 & 0.00006738 & 0.252686 & 0.00004468 & 0.356445 & 0.00003406 \\
	8      & 0.028076 & 0.00007690 & 0.289307 & 0.00005078 & 0.407715 & 0.00003894 \\
	9      & 0.031738 & 0.00008655 & 0.325928 & 0.00005713 & 0.457764 & 0.00004395 \\
	10     & 0.0354   & 0.00009619 & 0.363769 & 0.00006335 & 0.509033 & 0.00004895 \\
	11     & 0.039062 & 0.00010600 & 0.39917  & 0.00006982 & 0.560303 & 0.00005371 \\
	12     & 0.042725 & 0.00011600 & 0.435791 & 0.00007617 & 0.610352 & 0.00005872 \\
	13     & 0.046387 & 0.00012500 & 0.472412 & 0.00008264 & 0.661621 & 0.00006360 \\
	14     & 0.05127  & 0.00013500 & 0.507812 & 0.00008911 & 0.712891 & 0.00006860 \\
	15     & 0.054932 & 0.00014400 & 0.544434 & 0.00009521 & 0.762939 & 0.00007349 \\
	16     & 0.058594 & 0.00015400 & 0.581055 & 0.00010200 & 0.81543  & 0.00007825 \\
	17     & 0.063477 & 0.00016300 & 0.618896 & 0.00010800 & 0.865478 & 0.00008325 \\
	18     & 0.067139 & 0.00017300 & 0.655518 & 0.00011400 & 0.916748 & 0.00008826 \\
	19     & 0.072021 & 0.00018200 & 0.690918 & 0.00012100 & 0.966797 & 0.00009326 \\
	20     & 0.075684 & 0.00019200 & 0.727539 & 0.00012700 & 1.018066 & 0.00009814 \\
	21     & 0.080566 & 0.00020200 & 0.762939 & 0.00013300 & 1.068115 & 0.00010300 \\
	22     & 0.084229 & 0.00021100 & 0.79956  & 0.00014000 & 1.120605 & 0.00010800 \\
	23     & 0.089111 & 0.00022100 & 0.836182 & 0.00014600 & 1.170654 & 0.00011300 \\
	24     & 0.093994 & 0.00023000 & 0.872803 & 0.00015300 & 1.221924 & 0.00011800 \\
	25     & 0.097656 & 0.00024000 & 0.909424 & 0.00015900 & 1.273193 & 0.00012300 \\
	26     & 0.102539 & 0.00025000 & 0.946045 & 0.00016500 & 1.324463 & 0.00012700 \\
	27     & 0.107422 & 0.00025900 & 0.982666 & 0.00017200 & 1.375732 & 0.00013200 \\
	28     & 0.111084 & 0.00026900 & 1.018066 & 0.00017800 & 1.425781 & 0.00013700 \\
	29     & 0.114746 & 0.00027800 & 1.054687 & 0.00018400 & 1.478271 & 0.00014200 \\
	30     & 0.119629 & 0.00028800 & 1.091308 & 0.00019100 & 1.52832  & 0.00014700 \\
  	31     & 0.123291 & 0.00029800 & 1.12793  & 0.00019700 & 1.57959  & 0.00015200 \\
	32     & 0.128174 & 0.00030700 & 1.164551 & 0.00020300 & 1.629639 & 0.00015700 \\
 	33     & 0.135498 & 0.00031600 & 1.199951 & 0.00021000 & 1.680908 & 0.00016200 \\
 	34     & 0.140381 & 0.00032600 & 1.237793 & 0.00021600 & 1.733398 & 0.00016700 \\
 	35     & 0.144043 & 0.00033600 & 1.274414 & 0.00022300 & 1.784668 & 0.00017200 \\
 	36     & 0.150146 & 0.00034500 & 1.311035 & 0.00022900 & 1.834717 & 0.00017700 \\
 	37     & 0.155029 & 0.00035400 & 1.347656 & 0.00023500 & 1.887207 & 0.00018100 \\
	38     & 0.159912 & 0.00036400 & 1.383056 & 0.00024200 & 1.937256 & 0.00018600 \\
 	39     & 0.166016 & 0.00037300 & 1.420898 & 0.00024800 & 1.987304 & 0.00019100 \\
	40     & 0.170898 & 0.00038300 & 1.456299 & 0.00025400 & 2.039795 & 0.00019600 \\
	41     & 0.175781 & 0.00039200 & 1.49292  & 0.00026100 & 2.091064 & 0.00020100 \\
	42     & 0.180664 & 0.00040200 & 1.530762 & 0.00026700 & 2.141113 & 0.00020600 \\
	43     & 0.184326 & 0.00041200 & 1.566162 & 0.00027300 & 2.192383 & 0.00021100 \\
	44     & 0.189209 & 0.00042100 & 1.602783 & 0.00028000 & 2.243652 & 0.00021500 \\
	45     & 0.194092 & 0.00043000 & 1.638183 & 0.00028600 & 2.294922 & 0.00022000 \\
	46     & 0.197754 & 0.00044000 & 1.676025 & 0.00029200 & 2.346191 & 0.00022500 \\
	47     & 0.201416 & 0.00045000 & 1.711426 & 0.00029900 & 2.39624  & 0.00023000 \\
	48     & 0.20752  & 0.00045900 & 1.749267 & 0.00030500 & 2.44751  & 0.00023500 \\
	49     & 0.213623 & 0.00046800 & 1.784668 & 0.00031200 & 2.497558 & 0.00024000 \\
	50     & 0.218506 & 0.00047800 & 1.821289 & 0.00031800 & 2.548828 & 0.00024500 \\
\hline
  \end{tabular}
  }
  \end{table}

\begin{table}[h]
  \centering
    \caption{可変抵抗器(3-1端子間)の諸特性値}
    \label{tab:ER3-1}
     \scalebox{0.8}{
    \begin{tabular}{ccccccc}
    \hline
カウンタ変数 & Aでの$V_{U}$[\rm{V}] & Aでの$I_{U}$[\rm{A}] & Bでの$V_{U}$[\rm{V}] & Bでの$I_{U}$[\rm{A}]   & Cでの$V_{U}$[\rm{V}] & Cでの$I_{U}$[\rm{A}]\\
\hline
  0         & 0.007324       & 0.00000000 & 0.00732400 & 0.00000012 & 0.00732400 & 0.00000012 \\
  1         & 0.050049       & 0.00000488 & 0.04882800 & 0.00000513 & 0.05127000 & 0.00000500 \\
  2         & 0.101318       & 0.00000964 & 0.10009800 & 0.00000964 & 0.10376000 & 0.00000940 \\
  3         & 0.151367       & 0.00001465 & 0.15136700 & 0.00001465 & 0.15258800 & 0.00001453 \\
  4         & 0.202637       & 0.00001953 & 0.20263700 & 0.00001965 & 0.20629900 & 0.00001892 \\
  5         & 0.252686       & 0.00002454 & 0.25390600 & 0.00002441 & 0.25634800 & 0.00002429 \\
  6         & 0.305176       & 0.00002917 & 0.30395500 & 0.00002942 & 0.31005900 & 0.00002856 \\
  7         & 0.355225       & 0.00003394 & 0.35522500 & 0.00003418 & 0.36010700 & 0.00003381 \\
  8         & 0.406494       & 0.00003882 & 0.40649400 & 0.00003906 & 0.41259800 & 0.00003845 \\
  9         & 0.456543       & 0.00004407 & 0.45654300 & 0.00004395 & 0.46264600 & 0.00004358 \\
  10        & 0.507812       & 0.00004895 & 0.50781200 & 0.00004919 & 0.51757800 & 0.00004810 \\
  11        & 0.559082       & 0.00005396 & 0.55786100 & 0.00005420 & 0.56762700 & 0.00005310 \\
  12        & 0.610352       & 0.00005859 & 0.60913100 & 0.00005884 & 0.62011700 & 0.00005774 \\
  13        & 0.661621       & 0.00006348 & 0.66162100 & 0.00006360 & 0.67138700 & 0.00006274 \\
  14        & 0.71167        & 0.00006885 & 0.71167000 & 0.00006860 & 0.72387700 & 0.00006738 \\
  15        & 0.762939       & 0.00007336 & 0.76293900 & 0.00007349 & 0.77514600 & 0.00007227 \\
  16        & 0.81543        & 0.00007825 & 0.81298800 & 0.00007849 & 0.82763700 & 0.00007727 \\
  17        & 0.865478       & 0.00008325 & 0.86425800 & 0.00008350 & 0.87890600 & 0.00008191 \\
  18        & 0.916748       & 0.00008826 & 0.91552700 & 0.00008826 & 0.93017600 & 0.00008691 \\
  19        & 0.966797       & 0.00009326 & 0.96557600 & 0.00009302 & 0.98266600 & 0.00009155 \\
  20        & 1.016846       & 0.00009814 & 1.01684600 & 0.00009814 & 1.03393500 & 0.00009631 \\
  21        & 1.069336       & 0.00010300 & 1.06811500 & 0.00010300 & 1.08520500 & 0.00010100 \\
  22        & 1.120605       & 0.00010800 & 1.11938500 & 0.00010800 & 1.13647400 & 0.00010600 \\
  23        & 1.170654       & 0.00011300 & 1.17065400 & 0.00011300 & 1.18896500 & 0.00011100 \\
  24        & 1.221924       & 0.00011800 & 1.22070300 & 0.00011800 & 1.24023400 & 0.00011600 \\
  25        & 1.273193       & 0.00012300 & 1.27319300 & 0.00012300 & 1.29272400 & 0.00012000 \\
  26        & 1.324463       & 0.00012700 & 1.32324200 & 0.00012800 & 1.34399400 & 0.00012600 \\
  27        & 1.375732       & 0.00013200 & 1.37329100 & 0.00013200 & 1.39648400 & 0.00013000 \\
  28        & 1.427002       & 0.00013700 & 1.42334000 & 0.00013800 & 1.44897400 & 0.00013500 \\
  29        & 1.478271       & 0.00014200 & 1.47583000 & 0.00014200 & 1.49902300 & 0.00014000 \\
  30        & 1.52832        & 0.00014700 & 1.52587900 & 0.00014700 & 1.55273400 & 0.00014500 \\
  31        & 1.57959        & 0.00015200 & 1.57836900 & 0.00015200 & 1.60400400 & 0.00015000 \\
  32        & 1.629639       & 0.00015700 & 1.62841800 & 0.00015700 & 1.65649400 & 0.00015400 \\
  33        & 1.682129       & 0.00016200 & 1.67968700 & 0.00016200 & 1.70654300 & 0.00015900 \\
  34        & 1.733398       & 0.00016700 & 1.73095700 & 0.00016700 & 1.75903300 & 0.00016400 \\
  35        & 1.783447       & 0.00017200 & 1.78100600 & 0.00017200 & 1.81152300 & 0.00016900 \\
  36        & 1.834717       & 0.00017700 & 1.83349600 & 0.00017700 & 1.86279300 & 0.00017400 \\
  37        & 1.887207       & 0.00018100 & 1.88354500 & 0.00018200 & 1.91528300 & 0.00017800 \\
  38        & 1.937256       & 0.00018600 & 1.93481400 & 0.00018700 & 1.96533200 & 0.00018400 \\
  39        & 1.987304       & 0.00019100 & 1.98486300 & 0.00019200 & 2.01782200 & 0.00018800 \\
  40        & 2.039795       & 0.00019600 & 2.03613300 & 0.00019700 & 2.07153300 & 0.00019300 \\
  41        & 2.089844       & 0.00020100 & 2.08618100 & 0.00020200 & 2.12280300 & 0.00019800 \\
  42        & 2.141113       & 0.00020600 & 2.13745100 & 0.00020600 & 2.17407200 & 0.00020300 \\
  43        & 2.192383       & 0.00021100 & 2.18994100 & 0.00021100 & 2.22656200 & 0.00020700 \\
  44        & 2.243652       & 0.00021500 & 2.23999000 & 0.00021600 & 2.27661100 & 0.00021200 \\
  45        & 2.293701       & 0.00022100 & 2.29126000 & 0.00022100 & 2.32910100 & 0.00021700 \\
  46        & 2.34497        & 0.00022500 & 2.34130800 & 0.00022600 & 2.38037100 & 0.00022200 \\
  47        & 2.39624        & 0.00023000 & 2.39379900 & 0.00023100 & 2.43164000 & 0.00022700 \\
  48        & 2.44751        & 0.00023500 & 2.44384700 & 0.00023600 & 2.48535100 & 0.00023100 \\
  49        & 2.497558       & 0.00024000 & 2.49511700 & 0.00024000 & 2.53662100 & 0.00023600 \\
  50        & 2.547607       & 0.00024500 & 2.54638600 & 0.00024500 & 2.58789000 & 0.00024100 \\
\hline
\end{tabular}
    }
\end{table}

\begin{figure}[h]
  %\begin{minipage}[c]{0.5\hsize}
    \centering
    \subfloat{\includegraphics[scale=0.315]{./fig/3-2-1.pdf}}
	\caption{可変抵抗器(1-2端子間)の電圧電流特性}
	\label{fig:3-2-1}
	%\end{minipage}
  %\begin{minipage}[c]{0.5\hsize}
    \centering
   \subfloat{ \includegraphics[scale=0.315]{./fig/3-2-2.pdf}}
	\caption{可変抵抗器(2-3端子間)の電圧電流特性}
	\label{fig:3-2-2}
  %\end{minipage}
  %\begin{minipage}[c]{0.5\hsize}
  \centering
  \centering
        \subfloat{\includegraphics[scale=0.31]{./fig/3-2-3.pdf}}
	\caption{可変抵抗器(3-1端子間)の電圧電流特性}
	\label{fig:3-2-3}
 %\end{minipage}
\end{figure}
\clearpage
\subsection{実習3-3 CdSセンサの電圧電流特性}
\begin{itemize}
	\item プログラムは上記の\ref{ex32-block}と同じものを利用し,$R_{0}$の値は10\,k\rm{$\Omega$}とした.
	\item 実行後,フロントパネルは\wfig{ex33-flont}のようになり,計測結果は\wtab{CDS}のようになった.
	\item 計測データを基に作成した電圧電流特性のグラフは\wfig{3-3}である.
\end{itemize}

 \begin{figure}[h]
  \centering
\includegraphics[scale=0.35]{./fig/ex33-flont.pdf}
\caption{CdSセンサの電圧電流特性測定時のフロントパネル}
\label{fig:ex33-flont}
\end{figure}

\begin{figure}[h]
\centering
\includegraphics[scale=0.38]{./fig/3-3.pdf}
\caption{CdSセンサの電圧電流特性}
\label{fig:3-3}
\end{figure}

\begin{table}[h]
    \centering
    \caption{CdSセンサの諸特性値}
    \label{tab:CDS}
     \scalebox{0.7}{
    \begin{tabular}{ccccc}
    \hline
    カウンタ変数 & 自然状態での$V_{U}$[\rm{V}] & 自然状態での$I_{U}$[\rm{A}]  & ライト照射時の$V_{U}$[\rm{V}] &ライト照射時の$I_{U}$[\rm{V}]   \\
    \hline
0      & 0.007324 & 0        & 0.007324 & 0.0000000 \\
1      & 0.007324 & 0.000009 & 0.006104 & 0.0000092 \\
2      & 0.014648 & 0.000018 & 0.006104 & 0.0000189 \\
3      & 0.023193 & 0.000027 & 0.008545 & 0.0000289 \\
4      & 0.031738 & 0.000037 & 0.010986 & 0.0000387 \\
5      & 0.040283 & 0.000046 & 0.014648 & 0.0000483 \\
6      & 0.050049 & 0.000055 & 0.018311 & 0.0000580 \\
7      & 0.058594 & 0.000064 & 0.021973 & 0.0000675 \\
8      & 0.067139 & 0.000073 & 0.024414 & 0.0000774 \\
9      & 0.074463 & 0.000082 & 0.028076 & 0.0000868 \\
10     & 0.083008 & 0.000092 & 0.031738 & 0.0000966 \\
11     & 0.091553 & 0.0001   & 0.0354   & 0.0001060 \\
12     & 0.100098 & 0.00011  & 0.039062 & 0.0001160 \\
13     & 0.109863 & 0.000119 & 0.042725 & 0.0001250 \\
14     & 0.118408 & 0.000128 & 0.046387 & 0.0001350 \\
15     & 0.126953 & 0.000137 & 0.050049 & 0.0001450 \\
16     & 0.135498 & 0.000146 & 0.053711 & 0.0001550 \\
17     & 0.144043 & 0.000155 & 0.057373 & 0.0001640 \\
18     & 0.151367 & 0.000165 & 0.061035 & 0.0001740 \\
19     & 0.161133 & 0.000174 & 0.063477 & 0.0001830 \\
20     & 0.169678 & 0.000183 & 0.068359 & 0.0001930 \\
21     & 0.178223 & 0.000192 & 0.070801 & 0.0002030 \\
22     & 0.186768 & 0.000201 & 0.073242 & 0.0002130 \\
23     & 0.195312 & 0.00021  & 0.078125 & 0.0002220 \\
24     & 0.203857 & 0.000219 & 0.080566 & 0.0002320 \\
25     & 0.212402 & 0.000229 & 0.084229 & 0.0002410 \\
26     & 0.220947 & 0.000238 & 0.087891 & 0.0002510 \\
27     & 0.229492 & 0.000247 & 0.091553 & 0.0002610 \\
28     & 0.238037 & 0.000256 & 0.093994 & 0.0002710 \\
29     & 0.246582 & 0.000265 & 0.097656 & 0.0002800 \\
30     & 0.255127 & 0.000274 & 0.100098 & 0.0002900 \\
31     & 0.263672 & 0.000284 & 0.10376  & 0.0003000 \\
32     & 0.272217 & 0.000292 & 0.106201 & 0.0003090 \\
33     & 0.280762 & 0.000302 & 0.108643 & 0.0003190 \\
34     & 0.289307 & 0.000311 & 0.112305 & 0.0003290 \\
35     & 0.297852 & 0.00032  & 0.115967 & 0.0003380 \\
36     & 0.306396 & 0.000329 & 0.118408 & 0.0003480 \\
37     & 0.316162 & 0.000338 & 0.123291 & 0.0003580 \\
38     & 0.324707 & 0.000348 & 0.126953 & 0.0003670 \\
39     & 0.333252 & 0.000357 & 0.130615 & 0.0003770 \\
40     & 0.341797 & 0.000366 & 0.133057 & 0.0003870 \\
41     & 0.350342 & 0.000375 & 0.136719 & 0.0003960 \\
42     & 0.358887 & 0.000384 & 0.140381 & 0.0004060 \\
43     & 0.367432 & 0.000393 & 0.144043 & 0.0004160 \\
44     & 0.375977 & 0.000402 & 0.147705 & 0.0004250 \\
45     & 0.384521 & 0.000411 & 0.151367 & 0.0004350 \\
46     & 0.393066 & 0.000421 & 0.15625  & 0.0004440 \\
47     & 0.401611 & 0.00043  & 0.159912 & 0.0004540 \\
48     & 0.410156 & 0.000439 & 0.163574 & 0.0004640 \\
49     & 0.418701 & 0.000448 & 0.167236 & 0.0004730 \\
50     & 0.427246 & 0.000457 & 0.170898 & 0.0004830 \\
\hline
  \end{tabular}
  }
 \end{table}
\clearpage
\subsection{実習3-4 力センサの電圧電流特性}
\begin{itemize}
	\item プログラムは上記の\ref{ex32-block}と同じものを利用し,$R_{0}$の値は10\,k\rm{$\Omega$}とした.
	\item 計測結果は\wtab{PS}のようになった.
	\item 計測データを基に作成した電圧電流特性のグラフは\wfig{3-4}である.
\end{itemize}

\begin{table}[h]
  \centering
    \caption{力センサの諸特性値}
    \label{tab:PS}
     \scalebox{0.8}{
    \begin{tabular}{ccccc}
    \hline
    カウンタ変数 & 自然状態での$V_{U}$[\rm{V}] & 自然状態での$I_{U}$[\rm{A}]  & 圧力ありでの$V_{U}$[\rm{V}] &圧力ありでの$I_{U}$[\rm{A}] \\
    \hline
 0  & 0.007324 & 0.0000001  & 0.006104 & 0.000000122 \\
1  & 0.100098 & -0.0000001 & 0.019531 & 0.00000793 \\
2  & 0.198975 & 0.0000000  & 0.037842 & 0.0000157 \\
3  & 0.299072 & -0.0000002 & 0.058594 & 0.0000237 \\
4  & 0.397949 & -0.0000001 & 0.081787 & 0.0000316 \\
5  & 0.494385 & 0.0000005  & 0.102539 & 0.0000397 \\
6  & 0.593262 & 0.0000004  & 0.12085  & 0.0000477 \\
7  & 0.692139 & 0.0000005  & 0.136719 & 0.0000559 \\
8  & 0.789795 & 0.0000007  & 0.161133 & 0.0000636 \\
9  & 0.887451 & 0.0000011  & 0.177002 & 0.0000720 \\
10 & 0.987549 & 0.0000010  & 0.194092 & 0.0000804 \\
11 & 1.086426 & 0.0000011  & 0.218506 & 0.0000880 \\
12 & 1.185303 & 0.0000012  & 0.236816 & 0.0000959 \\
13 & 1.28418  & 0.0000015  & 0.253906 & 0.000104 \\
14 & 1.381836 & 0.0000015  & 0.269775 & 0.000113 \\
15 & 1.483154 & 0.0000016  & 0.289307 & 0.000121 \\
16 & 1.58081  & 0.0000017  & 0.307617 & 0.000129 \\
17 & 1.678467 & 0.0000020  & 0.323486 & 0.000137 \\
18 & 1.779785 & 0.0000020  & 0.343018 & 0.000146 \\
19 & 1.876221 & 0.0000022  & 0.357666 & 0.000154 \\
20 & 1.975097 & 0.0000022  & 0.371094 & 0.000163 \\
21 & 2.075195 & 0.0000022  & 0.396728 & 0.00017  \\
22 & 2.172851 & 0.0000026  & 0.411377 & 0.000179 \\
23 & 2.272949 & 0.0000026  & 0.428467 & 0.000187 \\
24 & 2.370605 & 0.0000027  & 0.445557 & 0.000195 \\
25 & 2.470703 & 0.0000028  & 0.461426 & 0.000204 \\
26 & 2.568359 & 0.0000031  & 0.482178 & 0.000212 \\
27 & 2.667236 & 0.0000033  & 0.496826 & 0.00022  \\
28 & 2.766113 & 0.0000033  & 0.511475 & 0.000229 \\
29 & 2.86499  & 0.0000035  & 0.531006 & 0.000237 \\
30 & 2.963867 & 0.0000035  & 0.540771 & 0.000246 \\
31 & 3.063965 & 0.0000035  & 0.552978 & 0.000255 \\
32 & 3.161621 & 0.0000038  & 0.562744 & 0.000263 \\
33 & 3.261718 & 0.0000038  & 0.579834 & 0.000272 \\
34 & 3.360595 & 0.0000039  & 0.5896   & 0.000281 \\
35 & 3.459472 & 0.0000042  & 0.609131 & 0.000289 \\
36 & 3.558349 & 0.0000042  & 0.625    & 0.000297 \\
37 & 3.658447 & 0.0000042  & 0.635986 & 0.000306 \\
38 & 3.757324 & 0.0000045  & 0.653076 & 0.000314 \\
39 & 3.85498  & 0.0000046  & 0.664062 & 0.000324 \\
40 & 3.953857 & 0.0000046  & 0.679932 & 0.000332 \\
41 & 4.053955 & 0.0000045  & 0.690918 & 0.000341 \\
42 & 4.152832 & 0.0000048  & 0.716553 & 0.000348 \\
43 & 4.252929 & 0.0000049  & 0.725098 & 0.000358 \\
44 & 4.348144 & 0.0000051  & 0.738525 & 0.000366 \\
45 & 4.449462 & 0.0000050  & 0.756836 & 0.000374 \\
46 & 4.547119 & 0.0000055  & 0.770264 & 0.000383 \\
47 & 4.644775 & 0.0000057  & 0.793457 & 0.000391 \\
48 & 4.744873 & 0.0000056  & 0.814209 & 0.000398 \\
49 & 4.846191 & 0.0000054  & 0.837402 & 0.000406 \\
50 & 4.942626 & 0.0000056  & 0.863037 & 0.000414 \\
\hline
\end{tabular}
}
\end{table}

\begin{figure}[h]
\centering
\includegraphics[scale=0.45]{./fig/3-4.pdf}
\caption{力センサの電圧電流特性}
\label{fig:3-4}
\end{figure}
\clearpage
\subsection{実習3-5 発光ダイオードの電圧電流特性}
\begin{itemize}
	\item プログラムは上記の\ref{ex32-block}と同じものを利用し,$R_{0}$の値は1\,k\rm{$\Omega$}とした.
	\item 計測結果は\wtab{LED}のようになった.
	\item 計測データを基に作成した電圧電流特性のグラフは\wfig{3-5}である.
\end{itemize}

\begin{table}[h]
\centering
\caption{LEDの諸特性値}
\label{tab:LED}
\scalebox{0.65}{
\begin{tabular}{ccccccccc}
\hline
ポインタ変数 & 赤色での$V_{U}$[\rm{V}]   & 赤色での$I_{U}$[\rm{A}]    & 青色での$V_{U}$[\rm{V}]      & 青色での$I_{U}$[\rm{A}]      & 白色での$V_{U}$[\rm{V}]    & 白色での$I_{U}$[\rm{A}]         & 緑色での$V_{U}$[\rm{V}]         & 緑色での$I_{U}$[\rm{A}]     \\
\hline
0      & 0.00732400 & 0.00000000  & 0.00732400 & 0.00000000  & 0.00610400 & 0.00000122 & 0.00610400 & 0.00000122  \\
1      & 0.09887700 & 0.00000366  & 0.09887700 & 0.00000366  & 0.09643600 & 0.00000732 & 0.09765600 & 0.00000610  \\
2      & 0.19775400 & 0.00000122  & 0.19531200 & 0.00000488  & 0.19653300 & 0.00000366 & 0.19897500 & -0.00000122 \\
3      & 0.29663100 & 0.00000122  & 0.29663100 & 0.00000122  & 0.29663100 & 0.00000244 & 0.29785200 & 0.00000000  \\
4      & 0.39672800 & 0.00000244  & 0.39917000 & -0.00000244 & 0.39672800 & 0.00000122 & 0.40039100 & -0.00000244 \\
5      & 0.49682600 & 0.00000244  & 0.49926800 & -0.00000122 & 0.49682600 & 0.00000244 & 0.49926800 & -0.00000122 \\
6      & 0.59570300 & 0.00000244  & 0.59936500 & -0.00000366 & 0.59692400 & 0.00000244 & 0.59570300 & 0.00000244  \\
7      & 0.69580100 & 0.00000244  & 0.69824200 & 0.00000000  & 0.69580100 & 0.00000244 & 0.69702100 & 0.00000122  \\
8      & 0.79589800 & 0.00000244  & 0.79589800 & 0.00000244  & 0.79711900 & 0.00000000 & 0.79834000 & -0.00000122 \\
9      & 0.89721700 & 0.00000122  & 0.89721700 & 0.00000244  & 0.89599600 & 0.00000244 & 0.89721700 & -0.00000122 \\
10     & 0.99609400 & 0.00000244  & 0.99731400 & -0.00000122 & 0.99487300 & 0.00000488 & 0.99853500 & -0.00000122 \\
11     & 1.09741200 & 0.00000000  & 1.09741200 & -0.00000122 & 1.09619100 & 0.00000244 & 1.09619100 & 0.00000122  \\
12     & 1.19751000 & 0.00000000  & 1.19873000 & -0.00000122 & 1.19506800 & 0.00000366 & 1.19873000 & -0.00000122 \\
13     & 1.29882800 & -0.00000122 & 1.29760700 & 0.00000000  & 1.29760700 & 0.00000122 & 1.29638700 & 0.00000122  \\
14     & 1.39770500 & 0.00000000  & 1.39648400 & 0.00000366  & 1.39648400 & 0.00000122 & 1.39770500 & 0.00000122  \\
15     & 1.49536100 & 0.00000122  & 1.49658200 & 0.00000122  & 1.49658200 & 0.00000244 & 1.49780300 & 0.00000000  \\
16     & 1.57959000 & 0.00001831  & 1.59545900 & 0.00000488  & 1.59668000 & 0.00000244 & 1.59668000 & 0.00000244  \\
17     & 1.63696300 & 0.00005981  & 1.69799800 & 0.00000000  & 1.69677700 & 0.00000122 & 1.69677700 & 0.00000244  \\
18     & 1.67602500 & 0.00012200  & 1.79687500 & 0.00000244  & 1.79809600 & 0.00000000 & 1.79687500 & 0.00000000  \\
19     & 1.70288100 & 0.00019400  & 1.89697200 & -0.00000122 & 1.89697200 & 0.00000000 & 1.89697200 & 0.00000122  \\
20     & 1.72485300 & 0.00027200  & 1.99462900 & 0.00000366  & 1.99584900 & 0.00000244 & 1.99584900 & 0.00000244  \\
21     & 1.74438500 & 0.00035300  & 2.09472600 & 0.00000488  & 2.09594700 & 0.00000244 & 2.09472600 & 0.00000488  \\
22     & 1.76025400 & 0.00043700  & 2.19604500 & 0.00000244  & 2.19604500 & 0.00000244 & 2.19604500 & 0.00000122  \\
23     & 1.77734400 & 0.00051900  & 2.29614200 & 0.00000244  & 2.29492200 & 0.00000366 & 2.29370100 & 0.00000488  \\
24     & 1.79077100 & 0.00060700  & 2.39135700 & 0.00000610  & 2.38769500 & 0.00001099 & 2.38403300 & 0.00001465  \\
25     & 1.80542000 & 0.00069100  & 2.47070300 & 0.00002686  & 2.45239200 & 0.00004639 & 2.45727500 & 0.00004150  \\
26     & 1.81762700 & 0.00077900  & 2.52807600 & 0.00007080  & 2.49145500 & 0.00010700 & 2.50854500 & 0.00009033  \\
27     & 1.83227500 & 0.00086700  & 2.56835900 & 0.00013200  & 2.51586900 & 0.00018300 & 2.54272400 & 0.00015600  \\
28     & 1.84326200 & 0.00095500  & 2.60009700 & 0.00019900  & 2.53295900 & 0.00026500 & 2.56835900 & 0.00022900  \\
29     & 1.85546900 & 0.00104100  & 2.62451100 & 0.00027500  & 2.54760700 & 0.00035000 & 2.58789000 & 0.00031000  \\
30     & 1.86767600 & 0.00112900  & 2.64648400 & 0.00035000  & 2.55981400 & 0.00043900 & 2.60498000 & 0.00039400  \\
31     & 1.87988300 & 0.00121700  & 2.66479500 & 0.00043300  & 2.56958000 & 0.00052900 & 2.61840800 & 0.00048100  \\
32     & 1.89209000 & 0.00130200  & 2.68066400 & 0.00051600  & 2.57812500 & 0.00062000 & 2.63183600 & 0.00056600  \\
33     & 1.90307600 & 0.00139300  & 2.69531200 & 0.00060200  & 2.58789000 & 0.00071000 & 2.64282200 & 0.00065600  \\
34     & 1.91406200 & 0.00148300  & 2.70874000 & 0.00068800  & 2.59399400 & 0.00080400 & 2.65258800 & 0.00074500  \\
35     & 1.92626900 & 0.00157100  & 2.72094700 & 0.00077600  & 2.60253900 & 0.00089600 & 2.66235300 & 0.00083600  \\
36     & 1.93725600 & 0.00165800  & 2.73315400 & 0.00086400  & 2.60986300 & 0.00098900 & 2.66967700 & 0.00092900  \\
37     & 1.94946300 & 0.00174600  & 2.74292000 & 0.00095500  & 2.61596700 & 0.00108300 & 2.67822200 & 0.00102100  \\
38     & 1.96044900 & 0.00183500  & 2.75268500 & 0.00104600  & 2.62207000 & 0.00117700 & 2.68676700 & 0.00111200  \\
39     & 1.97143500 & 0.00192400  & 2.76367200 & 0.00113500  & 2.62817400 & 0.00127100 & 2.69287100 & 0.00120600  \\
40     & 1.98242200 & 0.00201300  & 2.77343700 & 0.00122700  & 2.63305600 & 0.00136500 & 2.70019500 & 0.00129600  \\
41     & 1.99340800 & 0.00210200  & 2.78198200 & 0.00131600  & 2.63916000 & 0.00145900 & 2.70629900 & 0.00139200  \\
42     & 2.00561500 & 0.00219000  & 2.79052700 & 0.00140700  & 2.64404300 & 0.00155300 & 2.71362300 & 0.00148400  \\
43     & 2.01660100 & 0.00227800  & 2.79785100 & 0.00150100  & 2.65014600 & 0.00164700 & 2.71850600 & 0.00158000  \\
44     & 2.02636700 & 0.00236800  & 2.80639600 & 0.00159100  & 2.65502900 & 0.00174200 & 2.72460900 & 0.00167100  \\
45     & 2.03857400 & 0.00245500  & 2.81494100 & 0.00168200  & 2.65991200 & 0.00183600 & 2.73071300 & 0.00176600  \\
46     & 2.04956000 & 0.00254400  & 2.82226500 & 0.00177600  & 2.66479500 & 0.00193100 & 2.73559500 & 0.00186200  \\
47     & 2.06054700 & 0.00263400  & 2.82836900 & 0.00186800  & 2.66845700 & 0.00202600 & 2.74169900 & 0.00195400  \\
48     & 2.07275400 & 0.00272200  & 2.83691400 & 0.00195900  & 2.67456000 & 0.00212000 & 2.74658200 & 0.00205000  \\
49     & 2.08251900 & 0.00281000  & 2.84301700 & 0.00205400  & 2.67822200 & 0.00221800 & 2.75146500 & 0.00214600  \\
50     & 2.09350600 & 0.00289900  & 2.84912100 & 0.00214600  & 2.68310500 & 0.00231200 & 2.75634700 & 0.00224000  \\
\hline
\end{tabular}
}
\end{table}

\begin{figure}[h]
 \centering
        \includegraphics[scale=0.45]{./fig/3-5.pdf}
	\caption{LEDの電圧電流特性}
	\label{fig:3-5}
\end{figure}
\clearpage
\section{考察}
\subsection{課題考察}
\begin{enumerate}[実習2-1:]
\item 理想の``平均値''と``標準偏差''はそれぞれどのような値かを理由とともに考察せよ.

平均値は全体の総和から個々のデータの値を算出するものであるため,\wtab{fiveV}のように,理想の測定結果は誤差が0となるようなものである.そのため,平均値は各データと等しくなるような値であり,標準偏差はデータのばらつきを表しているため,0となるものが理想である.
\item 出力電圧表示値(Analog Output に入力した値)と計測電圧値(Analog Input から出力された値)の関係をグラフにし,近似直線を求めよ.傾きと切片の(理想の)値を予想し実際の値と比較し考察せよ.

\wfig{RMSE}は測定データ(\wtab{syotokusei})をプロットしたグラフを近似曲線と共に示したものである.線形性を有した測定結果であったため,高い精度で近似していることがわかる.
およそ(2\,\rm{V},2\,\rm{V})のところにも測定点が存在しており,およそ(4\,\rm{V},4\,\rm{V})の点にもあるため,近似直線の傾きは1,切片が0である直線が当測定データの近似直線となるだろう.
実際に\weq{saisyou1}, \weq{saisyou2}を用いて近似直線の傾きと切片を導出し,$a=0.999666982$, $b=0.000610273$となった.

\begin{table}[h]
\centering
\caption{電圧変化時の諸特性}
\label{tab:syotokusei}
\scalebox{0.75}{
\begin{tabular}{ccccc}
\hline
カウンタ変数& 出力電圧[\rm{V}] & 計測電圧[\rm{V}] & 誤差[\rm{V}] & RMSE[\rm{V}] \\
\hline
0       & 0           & 0.008545    & -0.00855  & 0.002576  \\
1       & 0.5         & 0.496826    & 0.003174  & 0.000957  \\
2       & 1           & 0.997314    & 0.002686  & 0.00081   \\
3       & 1.5         & 1.496582    & 0.003418  & 0.001031  \\
4       & 2           & 1.998291    & 0.001709  & 0.000515  \\
5       & 2.5         & 2.50122     & -0.00122  & 0.000368  \\
6       & 3           & 2.999267    & 0.000733  & 0.000221  \\
7       & 3.5         & 3.499756    & 0.000244  & 0.0000736  \\
8       & 4           & 4.001464    & -0.00146  & 0.000441  \\
9       & 4.5         & 4.499511    & 0.000489  & 0.000147  \\
10      & 5           & 4.998779    & 0.001221  & 0.000368 \\
\hline
\end{tabular}
}
\end{table}

\begin{figure}[h]
\centering
\includegraphics[scale=0.45]{./fig/RMSE.pdf}
\caption{出力電圧-計測電圧及び近似曲線}
\label{fig:RMSE}
\end{figure}
\end{enumerate}

\newpage
\begin{enumerate}[実習3-1:]
\item $R_{0}$の値を変更した時のそれぞれの電圧電流特性の傾きから抵抗値を求め.公称値と比較せよ.
また$R_{0}$の変化が算出した抵抗値にどのような影響を及ぼすのかについて考察せよ.

\wtab{sqr-R0}に,それぞれの$R_{0}$で電圧-電流のグラフ(x:電圧, y:電流)の近似直線をExcel上で最小二乗法により算出した.
傾きは\weq{1/R}で与えられるため,傾きの値の逆数をとることにより,抵抗値を求めることができる.(同表での計算値列)\\
どの$R_{0}$でも計算値の方が公称値より大きくなっていることがわかる.しかし,抵抗値によって誤差が変わり,最も誤差の大きい10\,k\rm{$\Omega$}と最小の100\,k\rm{$\Omega$}ではおよそ266\,倍もの違いがある.
そのため,測定素子や手法などによって適切な抵抗を利用する必要があるだろう.

\begin{equation}
\frac{I}{V}=\frac{1}{R}
\label{eq:1/R}
\end{equation}

\begin{table}[h]
\centering
\caption{$R_{0}$の公称値と計算値}
\label{tab:sqr-R0}
\begin{tabular}{cccc}
\hline
公称値[\rm{$\Omega$}]  & 近似直線の傾き[\rm{1/$\Omega$}]   & 計算値[\rm{$\Omega$}]  &誤差[\rm{$\Omega$}] \\
\hline
100  & 0.00999947  & 100.005298 & 0.005298412 \\
1k   & 0.000999971 & 1000.0287 & 0.028702093 \\
10k  & 0.00010     & 10000.8006 & 0.800641642 \\
100k & 0.00001     & 100000.003 & 0.003010618 \\
\hline
\end{tabular}
\end{table}
\item 各端子間のつまみ位置による電圧電流特性の変化の仕方から,使用した可変抵抗器の内部構造を予測せよ.

まず,\wfig{3-2-3}より,端子3-1間はつまみ位置に依存せず10\,k\rm{$\Omega$}となっている.(\wtab{3-1R})
次に端子1-2間は,つまみをA$\to$B$\to$Cと変化させていくと,電流が流れにくくなる.つまり,抵抗値が増加している.
また,端子2-3間は1-2間での動作と逆の動きであるため,\wfig{build}のような構造であると考えられ,\wfig{ALPS}との一致する.

\begin{table}[h]
\centering
\caption{計測値より導出される端子3-1間の抵抗値}
\label{tab:3-1R}
\begin{tabular}{cccc}
\hline
接続先&A&B&C\\
\hline
電圧[\rm{V}] & 10403.07355 &10365.33156 & 10732.50484\\
\hline
\end{tabular}
\end{table}

\begin{figure}[h]
\centering
\includegraphics[scale=0.5]{./fig/build.pdf}
\caption{可変抵抗器の構造予測}
\label{fig:build}
\end{figure}

\item  電圧電流特性の変化から,CdSセンサの抵抗値と光強度の関係について考察せよ.またその(CdS セルの)原理を調査し,実験結果が正しいか確認せよ.

\wfig{3-3}より,光強度が増加すると,電流が大きくなる.つまり,抵抗値が小さくなっていることが読み取れる.この結果は原理と一致しているため,正しい結果だといえる.(参照:\ref{CDSG})

\item 電圧電流特性の変化から,力の強さに依存して力センサの抵抗値がどのように変化するか考察せよ.またその(感圧センサの)原理を調査し,実験結果が正しいか確認せよ.

\wfig{3-4}より,力が増加すると電流も増加すること,抵抗値が減少することがわかる.このことは\ref{PG}と整合性があるため,正しい結果である.

\item 発光の色が異なるということは物理量として何が異なっているのかを示し,LEDの色に応じて電圧電流特性の特徴(立ち上がり電圧,傾き)どのような違いがあるか考察せよ.

発色の色の違いは``波長''である.
参考資料として\wfig{rl}に色と波長の関係を示す.\\
\wfig{3-5}から色により,立ち上がり電圧(Forward Voltage Drop,$V_{F}$)が異なり,赤色が他の色より$V_{F}$の値が小さいことがわかる.
また,傾きも青色,白色,緑色が同じような傾き(およそ0.0005\,1/\rm{$\Omega$})をとっているのに対し,赤色はそれらより傾きが大きい値(およそ0.002\,1/\rm{$\Omega$})である.\\
ここで,光のエネルギー$W$は\weq{W}で与えられる\cite{1130000795538269056}.つまり,光の波長が長くなれば(赤外線に近づけば),エネルギーは減少するということを意味し,今回の結果はこの方程式と矛盾はない.\\
すなわち,LEDの立ち上がり電圧は赤外線に近づくほど小さくなり,傾きは赤外線に近づくほど急になるといえる.\\
また,白色LEDは\ref{LEDG}で述べたように,青色と黄色蛍光体を用いる手法と光の3原色を用いる手法があるが,今回使用したLEDはグラフ上で,赤色と緑色の中間に位置しており,可視光の中間点であるおよそ$525\,\rm{nm}$(\weq{half})に近いため,後者の手法で作成されたものだと考えられる.
\begin{align}
W=h\nu&=h\frac{c}{\lambda}\label{eq:W} \\
h:プランク定数&, \nu:振動数, c:光速\nonumber
\end{align}
\begin{equation}
\label{eq:half}
(650+400)/2=525\,\rm{nm}
\end{equation}

\begin{figure}[h]
\begin{minipage}[c]{0.5\hsize}
\centering
\includegraphics[scale=0.5]{./fig/波長による光の色の変化.png}
\caption{光と波長\cite{gbcalasdjdv}}
\label{fig:rl}
  \end{minipage}
  \hfill
  \begin{minipage}[c]{0.5\hsize}
    \centering
\caption{発色光の違いと素子の特徴}
\label{tab:}
\begin{tabular}{ccc}
\hline
発色光&傾き[\rm{1/$\Omega$}]&$V_{F}$[\rm{V}]\\
\hline
赤色 & 0.00118  & 15 \\
青色 & 0.000501 & 18 \\
白色 & 0.00056  & 20 \\
緑色 & 0.000533 & 19 \\
\hline
\end{tabular}
\end{minipage}
\end{figure}
\end{enumerate}

\subsection{独自考察}
\begin{itemize}
\item 0\,\rm{V}で相対誤差が他の電圧測定時と異なり誤差が大きくなったのは,実際に実験をする際にノイズなどの影響により,理論通りに0\,\rm{V}にならかったためだと思われる.よって0\,\rm{V}近傍の計測では,遅延時間の設定が他の測定電圧より長くとる必要があるのではないだろうか.
\end{itemize}

\clearpage
\section{結論}
本実験を通して,
\begin{itemize}
	\item 自動計測の基礎をLabVIEWおよびMyRIOを使用して方法の習得.
	\item 最小二乗法などの近年でも有用な数値処理について計算方法の習得.
	\item 間接測定を利用した,測定データから抵抗値の導出.
	\item さまざまな素子の電流電圧特性の確認.
\end{itemize}
を達成することができた.

\newpage
\printbibliography[title=参考文献]
\end{document}
