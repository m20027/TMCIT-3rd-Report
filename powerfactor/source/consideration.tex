\clearpage
\section{考察}
\begin{enumerate}[1.)]
	\item 各設定に対して,横軸が電流,縦軸が電力のグラフ(一つにまとめる)を描け.
	
	\wfig{L}に$X_{L}$での電流-電力特性を,\wfig{C}に$X_{C}$での電流-電力特性を示す.
	どちらの場合においても,力率が$1.0$に近づくほど全体的に多くの電力が得られていることがわかる.また,力率による電力への影響は電流が大きい場合であるほど大きくなっている.
	\begin{figure}[h]
	\centering
	\includegraphics[scale=0.6]{./data/L/L.pdf}
	\caption{$X_L$の電流-電力特性}
	\label{fig:L}
	\end{figure}
	\begin{figure}[h]
	\centering
	\includegraphics[scale=0.6]{./data/C/C.pdf}
	\caption{$X_C$の電流-電力特性}
	\label{fig:C}
	\end{figure}
	\item 各電流計の指示に対して,横軸が力率,縦軸が電力のグラフ(一つにまとめる)を描け.
	
	\wfig{},\wfig{}
	\item 電力と電圧,電流,力率の関係を述べよ.
	\item 
\end{enumerate}