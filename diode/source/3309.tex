\documentclass[11pt,dvipdfmx]{ujarticle}
\usepackage{eee,scalefnt,graphicx}

\bibliography{3rd_diode}
\begin{document}

\begin{jikkenTitle}
 \gakunen{3} 
 \numTitle{2}{ダイオードを用いた半導体の基本的性質の検証} 
 \subTitle{(Semiconductor's basic nature verification using diode)}
 \jikkenbi{令和04年06月02日(木)} 
 \jikkenbiII{令和04年06月09日(木)} 
 \kyoudou{3301青木柊人,3305市川 潤} 
 \kyoudouII{3317杉山 滉太,3326塚原 秀翔} 
 \yoteibi{06/16}
 \yoteibiII{06/23}
 \yoteibiIII{06/30}
 \hanNumberName{1}{3309}{大山 主朗} 
\end{jikkenTitle}

\section{目的}
今回の実験では以下の3点を目的とする.
\begin{itemize}
	\item ダイオードの原理を知り,実験により整流作用を理解することでダイオードを使用できるようにする.
	\item ツェナーダイオードの特性を学び,ツェナーダイオードを使用できるようにする.
	\item 太陽電池の特性を知り,使用できるようにする.
\end{itemize}

\clearpage

\section{原理}
\subsection{磁束(magnetic flux)\cite{113028227042}}
磁気量$q_{m}$の磁極からは$q_{m}$本の磁束が発生し,磁束は途切れたり枝分かれすることはない.
磁束に垂直な単位面積の面を貫く磁束を磁束密度(magnetic flux density)といい,磁束密度$\boldsymbol{B}$と磁界$\boldsymbol{H}$には
\begin{equation}
	\boldsymbol{B}=\mu\boldsymbol{H}\,[\rm{Wb/m^{2}}] \,or\, [\rm{T}] 
\end{equation}
という関係がある.

\subsection{磁気双極子(magnetic dipole)\cite{7652}}
正負の磁極の対.本質的には微小円電流で以下のように表される.
\begin{equation}
	\boldsymbol{m}=\mu_{0}IS\boldsymbol{n}
\end{equation}
ここで,$S$はループ面の面積,$\boldsymbol{n}$はループ面の単位法線ベクトルである.

\subsection{磁気モーメント(magnetic moment)\cite{7652}\cite{113028227162}}
磁気モーメントは以下のように表すことができる.
\begin{equation}
	\boldsymbol{\mu}=IS\boldsymbol{n}
\end{equation}

\subsection{磁化(magnetization)\cite{11302042}}
磁界に対して応答を示す物質を磁性体(magnetic material)という.
通常の磁性体には多数の磁気双極子が含まれている.単位体積に含まれる磁気モーメントの和を磁化と呼ぶ.
ここで薄い板状の磁性体を考える.板に垂直方向に一様な磁界がかかっており,外部における磁場を$H_{0}$とすると,磁性体の外部での磁束密度$B$は
\begin{equation}	
	B=\mu_{0}H_{0}\,[\rm{T}]
\end{equation}
という関係を満たす.一方,この場合には磁性体は板の厚み方向に磁化されており,その値を$M$とすると,磁性体の両面には単位面積あたり$\pm \mu_{0}M$の磁極が発生する.このとき磁極は磁性体内部に
$-M$という磁界を発生させる.これらより磁性体内部の磁束密度以下のようになる.
\begin{align}
	H&=H_{0}-M\nonumber\\
	\mu_{0}H&=\mu_{0}H_{0}-\mu_{0}M\nonumber\\
	\mu_{0}H&=B-\mu_{0}M\nonumber\\
	B&=\mu_{0}(H+M)\,[\rm{T}]
\end{align}
となる.向きも含めて考えると上式は\weq{mo}のようになる.
\begin{equation}
	\boldsymbol{B}=\mu_{0}(\boldsymbol{H}+\boldsymbol{M})\,[\rm{T}]
	\label{eq:mo}
\end{equation}
また,磁界と磁化の関係を\weq{mmo}のように表現する場合
\begin{equation}
	\boldsymbol{M}=\chi\boldsymbol{H}\,[\rm{T}]
	\label{eq:mmo}
\end{equation}
$\chi$を磁化率(単位は無次元)などと呼び,\weq{mo}に代入し,\weq{bh}の関係を用いると
\begin{equation}
	\boldsymbol{B}=\mu \boldsymbol{H}\,[\rm{T}]
	\label{eq:bh}
\end{equation}
\weq{mu}のような関係があることがわかる.
\begin{align}
	\boldsymbol{B}&=\mu_{0}(\boldsymbol{H}+\boldsymbol{M})\nonumber\\
	\mu \boldsymbol{H}&=\mu_{0}(\boldsymbol{H}+\chi\boldsymbol{H})\nonumber\\
	\mu &=\mu_{0}(1+\chi)\,[\rm{F/m}]\label{eq:mu}
\end{align}

\subsection{キュリーの法則(Curie's law)\cite{76972}}
常磁性物質において,磁化率と温度の関係(反比例)を示す法則で以下のように表すことができる.
\begin{equation}
	\chi=\frac{C}{T}\,[\rm{-}]
\end{equation}
ここで,$C$はキュリー定数$\,[\rm{K}]$,$T$は絶対温度$\,[\rm{K}]$

\subsection{磁気回路(magnetic circuit)}
\wfig{hys:jikikairo}に示すように断面積$S\,[\mathrm{m}^2]$,平均磁路長$L\,[\mathrm{m}]$の鉄心に巻数$N_1\,[\mathrm{Turn}]$のコイルを巻き,これに$I\,[\mathrm{A}]$の電流を流すと,起磁力$N_1\cdot I\,[\mathrm{A}\cdot\rm{Turn}]$を生じる.
この起磁力により
\begin{equation}
	\phi = \frac{N_1\cdot I}{R_m}
\end{equation}
の磁束$\phi\,[\mathrm{Wb}]$を生じる.ここで$R_m$は以下に示す磁気抵抗である.
\begin{equation}
	R_m= \frac{L}{\mu_0 \mu_s S}
\end{equation}
ただし,$\mu_0 = 4\pi\times 10^{-7}\,\mathrm{F/m}$ は真空の透磁率であり,$\mu_s$は鉄心の比透磁率である.
ここで,磁路1\,mあたりの起磁力を磁化力$H\,[\mathrm{A/m}]$という. 磁化力$H$は
\begin{equation}
	H=\frac{N_1\cdot I}{L}
\end{equation}
である.
\begin{figure}[htbp]
	\centering
	\includegraphics[width=70mm]{fig/magnetism_circuit.pdf}
	\caption{磁気回路}
	\label{fig:hys:jikikairo}
\end{figure}

鉄心の磁化力$H$と磁束密度$B$との関係を示す曲線をB-H曲線といい,一般に\wfig{hys:bhcurve}(a)のような飽和特性になる.
また磁化力$H$を正負の方向に増減すると,\wfig{hys:bhcurve}(b)の様なヒステリシス曲線(Hysteresis curve)になる.
\begin{figure}[htbp]
	\centering
	\begin{tabular}{cc}
		\includegraphics[width=70mm]{fig/bhcurve.pdf} &
		\includegraphics[width=70mm]{fig/hysteresis.pdf} \\
		(a) B-H 曲線 & (b) ヒステリシス曲線
	\end{tabular}
	\caption{B-H曲線とヒステリシス曲線}
	\label{fig:hys:bhcurve}
\end{figure}

\subsection{交流磁化特性}
\label{zika}
\wfig{hys:transformer}の変圧器のように,鉄心に巻かれた巻数$N_1$のコイルに交流電圧$V_1$を加えると,鉄心中に交番磁束$\dot{\phi}$を作るための電流(励磁電流)$i_0$が流れる.このとき磁束密度$B$と磁化力$H$との間にはヒステリシス特性があるため,励磁電流は\wfig{hys:hizumi}のようにひずみを生ずる.この現象を逆に利用して,励磁電流$i_0$と交番磁束$\dot{\phi}$の波形をなんらかの方法で取り出し,オシロスコープのX軸に励磁電流$i_0$の波形,Y軸に交番磁束$\dot{\phi}$の波形を入力すれば,オシロスコープの画面に鉄心のヒステリシス特性(B-H曲線)が描かれる.
\begin{figure}[htbp]
	\centering
	\includegraphics[width=140mm]{fig/transformer.pdf}
	\caption{変圧器の交流磁化特性測定回路}
	\label{fig:hys:transformer}
\end{figure}

励磁電流$i_0$の波形を直接取り出すのは難しいので,\wfig{hys:transformer}において励磁電流$i_0$が抵抗$R_h$を流れるときの電圧変化,すなわち
\begin{equation}
	V_h = i_0R_h
\end{equation}
として取り出す.また,交番磁束$\dot{\phi}$は次の様にして取り出す.

\wfig{hys:transformer}において二次巻線$N_2$と鎖交する磁束の時間に対する変化が二次誘起電圧$e_2$として現れるため
\begin{equation}
	e_2 = -N_2\frac{d \phi}{dt}
	\label{eq:hys:e2}
\end{equation}
となり,\weq{hys:e2}を変形すると
\begin{equation}
	d \phi = \frac{1}{N_2}\times e_2\times dt
	\label{eq:hys:dphi}
\end{equation}
となるから,交番磁束$\phi$は\weq{hys:dphi}を積分すれば求まることとなる.すなわち,二次巻線に発生する電圧$e_2$を時間で積分すればよい.そこで二次側にCR積分回路を接続しコンデンサCの両端から$e_2$を積分した,交番磁束に比例した電圧をとりだす.

\begin{figure}[h]
  \begin{minipage}[c]{0.5\hsize}
    \centering
    \includegraphics[scale=1.2]{fig/hizumi_a.pdf} 
    \caption{ヒステリシス現象のない場合}
  \end{minipage}\\
  \begin{minipage}[c]{0.5\hsize}
    \centering
    \includegraphics[scale=1.2]{fig/hizumi_b.pdf}
    \caption{ヒステリシス現象のある場合}
  \end{minipage}
  \centering
  \caption{ヒステリシス現象}
   \label{fig:hys:hizumi}
\end{figure}

\subsection{磁区(magnetic domain)\cite{7697152}}
\wfig{domain}にヒステリシスループと磁区の関係を示す.
ヒステリシスループにおいてループと縦軸が交わる場所での磁化値を残留磁化といい,ループと横軸が交わる場所での磁界の値を保磁力という.
強磁性体は外部磁界がなくても自発磁化を持っているが,全体が一様に磁化されていると外部の空間に磁界を発生し,その磁界によるエネルギーが余分に余ってしまう.
そのため,強磁性体は自らを磁化の向きが異なる区間を磁区といい,磁区どうしの境界面を磁壁という.
強磁性は磁気モーメントどうしにお互いに同じ向きを向うとする相互作用がはたらくことで出現する.磁壁の両側では異なる向きの磁気モーメントが対峙することになるので,磁壁では余分なエネルギーが発生する.
このエネルギーの増加と,外部に磁場を発生させることによるエネルギーの増加の合計を最も小さくするように磁区構造が決まる.
\begin{figure}[h]
	\centering
	\includegraphics[scale=0.35]{fig/domain.png}
	\caption{ヒステリシスループと磁区\cite{cite-keygsdfz}}
	\label{fig:domain}
\end{figure}

\subsection{磁気飽和現象\cite{xdrcfhgvjb}}\label{hohwa}
界磁電流が大きくなると,エアギャップ磁束が界磁電流に比例せず,頭打ちになる現象.
同期機の磁気回路(磁束の通路)は磁極,エアギャップ,電機子歯,電機子鉄心,界磁継鉄(円筒界磁機では回転子鉄心)からなっており,このうち,エアギャップ以外はケイ素鋼板や鋼材などの強磁性体である.強磁性体の磁気分極は磁界の強さに比例せず,ある値に漸近する磁気飽和現象を有する.\\
それゆえ,同期機の磁気回路では界磁電流の増加に伴い,エアギャップの磁気抵抗は一定であるが強磁性材料部の磁気抵抗が増大するため,磁束は飽和現象を呈する.磁気飽和の程度を表すのに飽和係数を用い,無負荷飽和曲線上の定格電圧に対して,次式で表される.
\begin{equation}	
	飽和係数=\frac{鋼材部分に必要な界磁電流}{エアギャップに必要な界磁電流}
\end{equation}

\subsection{ビオ-サバールの法則(Biot-Savart's law)\cite{1130282271626280192}}
\begin{equation}
	d\boldsymbol{B}=\frac{\mu_{0}}{4\pi}\frac{Id\boldsymbol{l}\times \boldsymbol{R}}{R^{3}}
\end{equation}

\subsection{アンペールの法則(Ampère's circuital law) \cite{12}}
積分形のアンペールの法則は以下で与えられる.
\begin{equation}
	\oint_{C} \boldsymbol{\boldsymbol{r}}\cdot d\boldsymbol{l}=\mu_{0}I
\end{equation}

\subsection{電磁誘導(electromagnetic induction)\cite{titech}}
電磁誘導の法則は,時間変化する磁場中において固定した回路にもたらされる起電力がその回路を貫く磁束の時間変化に比例しているというものである.
また,磁束というものは回路に関係なく任意の曲面において定義できる.
起電力は導体の2点間の電位差として定義されるが,それは電場の線積分として表現される.任意の閉じた経路$C$を考えると 
\begin{equation}
	\oint_{C=\partial S}d\boldsymbol{r}\cdot \boldsymbol{E}(\boldsymbol{r}, t)=-\frac{d}{dt}\int_{S}d\boldsymbol{S}(\boldsymbol{r})\cdot \boldsymbol{B}(\boldsymbol{r}, t)
\end{equation}
である.$S$は$C$を境界にもつ面領域を表す.
この領域は回路に関係なく任意だが,時間に依存せず固定されているとする.
$C$が回路に一致すれば左辺は起電力を表すが,回路でなくても電場はあるので意味をもつ.
これがFaradayの法則の積分形である.

また,\weq{sto}の関係より,

\begin{equation}
	\oint_{C}d\boldsymbol{r}\cdot \boldsymbol{E}(\boldsymbol{r})=0
	\label{eq:sto}
\end{equation}
\begin{equation}
\int_{S}d\boldsymbol{S}\cdot \left(\boldsymbol{\nabla} \times \boldsymbol{E}(\boldsymbol{r}, t)+\frac{\partial}{\partial t}\boldsymbol{B}(\boldsymbol{r}, t)\right)=0
\end{equation}

よって,微分形は以下のように示される.
\begin{equation}
\boldsymbol{\nabla} \times \boldsymbol{E}(\boldsymbol{r}, t)+\frac{\partial}{\partial t}\boldsymbol{B}(\boldsymbol{r}, t)=0
\end{equation}
この公式は磁場が時間変化する場所では電場もまた時間変化しながら存在していることを意味している.
また,電場が空間的に変動しておりその回転が有限であるとき,時間依存する磁場が存在すると考えることもできる.
静電場の場合,電場は渦をつくることができず電場の回転が$0$であったが,時間変化のある場合には時間依存する磁場があるので電場の回転は$0$にならない.
時間依存性を無視すると静電場の法則に帰着する.

\subsection{抵抗損と漂遊負荷損\cite{1130282271832577152}}
ブリッジ法などで測定した巻線の抵抗(直線抵抗)を$r_{d}$とし,巻線に流れる交流電流を$I$とすると,抵抗損は$I^{2}r_{d}$である.しかし,実際の損失はこの値より大きくなる.その理由は次の2つである.
\begin{enumerate}[(1)]
	\item \textbf{導体内のうず電流損}\\
	電流による漏れ磁束が導体自身の断面に鎖交するため,導体内にうず電流が発生し,電流密度が不均一になり,導体断面積が減少したのと同じ結果となり抵抗が増加する.またその増加分は$5 \sim 20\,\%$である.また,うず電流は抵抗に反比例するため,巻線の断面積が大きい場合はうず電流を低減することができる\cite{11302822718325772}\cite{1130282270467697152}.
	\item \textbf{構造材料内の損失}\\
	漏れ磁束の一部は,タンクの側板,締め付けボルトなどを通るため,それらの部分にうず電流損失やヒステリシス損失が生じる.以上の2つを合わせて漂遊負荷損といい,抵抗損の$5 \sim 20\,\%$になる.
	抵抗損と漂遊負荷損の和が負荷損であるが,その値はほとんど電流の2乗に比例する.すなわち,二次負荷電流を$I_{2}$とすれば負荷損$W_{l}$は
	\begin{equation}
		W_{l}=I_{2}^{2}r
	\end{equation}
	で表される.
\end{enumerate}

\subsection{積分回路}
\wfig{ad}に示すようなCR回路を積分回路という.
この回路で入力信号を積分した信号を取り出すことができる.
同図で$S_{1}-S_{2}$に電源$e=E\sin \omega t$なる電圧が印加されている場合を考える.
この時回路を流れる電流$I$は$R\gg 1/\omega C$ならば\footnote{この回路の合成インピーダンスが抵抗のみであると近似できる条件}キャパシタにかかる電圧を$E_{C}$とすると
\begin{equation}
	I=\frac{E_{C}}{R}\sin \omega t
	\label{eq:I}
\end{equation}
となる.このとき$E_{C}$は,コンデンサの電荷を$q$とすれば
\begin{equation}
	E_{C}=\frac{q}{C}
	\label{eq:qcv}
\end{equation}
と表せる.また,電荷$q$は電流を積分したものであることと,\weq{I}より
\begin{align}
	q&=\int I dt\nonumber \\
	&=\int \frac{E}{R}\sin \omega tdt
	\label{eq:int}
\end{align}
\weq{qcv}と\weq{int}より
\begin{equation}
	E_{C}=\int \frac{E}{RC} \sin \omega tdt
	\label{eq:sekibun}
\end{equation}
\weq{sekibun}より,コンデンサ$C$の両端電圧$E_{C}$は入力信号$e$を積分した信号に比例した信号が得られる.

\subsection{変圧器の原理\cite{jknv}}

\subsection{変圧器の構造}

\clearpage
\section{実験}
\subsection{ダイオードの特性実験}
\subsubsection{ダイオードの実験器具}
使用した実験器具を\wtab{kigu}に示す.
\begin{table}[h]
  \centering
  \caption{実験装置}
  \label{tab:kigu}
  \scalebox{1.0}{
  \begin{tabular}{cccccc}
    \hline
    機器名&製造元&型番&シリアル番号(または管理番号)\\
    \hline
    ダイオード&不明&1N4002&不明\\
    直流電源&YOKOGAWA&PA1811&L96-000668\\
    直流用電圧計&YOKOGAWA&YAS 1991&71 BA0 3371\\
    ミリアンペア直流用電流計&YOKOGAWA&YES 1990&70 BA0 1812\\
    マイクロアンペア直流用電流計&YOKOGAWA&B-5036.H1.10/10&B5036\\
    \hline
  \end{tabular}
}
\end{table}

\subsubsection{ダイオードの実験方法}
\begin{enumerate}[(1)]
	\item \wfig{bias}のように回路を構築した.なお,ダイオード$D$は1N4002を用いた.
	\item 順バイアス$E$を加え,電圧$V_{D}$(0から0.8\,\rm{V}まで0.1\,\rm{V}刻みで変化)と電流$I_{D}$を計測した.その際,電流計はミリアンペア計を利用し,端子は$300\,\rm{mA}$に接続した.
	\item 計測したデータをプロットし,データ数が不足していた$0.6\,\rm{V}$から$0.8\,\rm{V}$の区間は$V_{D}$を$0.025\,\rm{V}$刻みで計測した.
	\item 計測データを基に,$V_{D}-I_{D}$特性をグラフにまとめた.
	\item 回路を\wfig{bias}の回路でダイオード$D$を逆向きに接続するように変更し,逆バイアス$E$を$0\,\rm{V}$から$0.8\,\rm{V}$まで$1\,\rm{V}$刻みで加え,電流$V_{D}$と電流$I_{D}$を計測した.なお,電流はマイクロアンペア計で計測し,端子は$3\,\rm{mA}$に接続した.
	\item 順バイアスと同様に,$V_{D}-I_{D}$特性をグラフにまとめた.
	\begin{figure}[h]
	\centering
	\includegraphics[scale=0.65]{./fig/bias.pdf}
	\caption{順バイアス測定回路}
	\label{fig:bias}
	\end{figure}
\end{enumerate}

\subsubsection{ダイオードの結果}
\begin{itemize}
	\item 測定結果を\wtab{bias}に示す.$0.400\,\rm{V}$まではほぼ電流が流れなかったが,それ以降は電圧増加とともに電流も増加し,$0.650\,\rm{V}$付近から急激に増加していることがわかる.
このような変化は抵抗の変化の仕方と異なっている.
	\item 順バイアスを印加した際の計測データから作成したグラフを近似曲線とともに\wfig{vias-graph-n}に示す.
	\item 逆バイアスを印加した際のグラフを\wfig{bias-rev}に示す.
	\begin{table}[h]
	\centering
	\caption{順・逆バイアスにおける電圧電流の関係}
	\label{tab:bias}
	\scalebox{1.0}{
	\begin{tabular}{cccc}
	\hline
	電圧(順バイアス)$V_{D}$[\rm{V}]  &電流$I_{D}$[\rm{mA}]   & 電圧(逆バイアス)$V_{D}$[\rm{V}] &  電流$I_{D}$[$\mu$\rm{A}] \\
	\hline
	0.000 & 0.0   & 0 & 0 \\
	0.100 & 0.0   & 1 & 0 \\
	0.200 & 0.0   & 2 & 0 \\
	0.300 & 0.0   & 3 & 0 \\
	0.400 & 0.0   & 4 & 0 \\
	0.500 & 1.0   & 5 & 0 \\
	0.600 & 2.0   & 6 & 0 \\
	0.625 & 4.0   & 7 & 0 \\
	0.650 & 6.0   & 8 & 0 \\
	0.675 & 10.0  &  - &  - \\
	0.700 & 19.8  &  - &  - \\
	0.725 & 29.0  & -  & -  \\
	0.750 & 48.0  & -  & -  \\
	0.775 & 71.0  & -  &  - \\
	0.800 & 116.0 & -  &-  \\
	\hline
	\end{tabular}
	}
\end{table}

\begin{figure}[h]
\centering
\includegraphics[scale=0.65]{./data/diode/bias-n.pdf}
\caption{順バイアスの$V_{D}-I_{D}$特性}
\label{fig:vias-graph-n}
\end{figure}
\end{itemize}

\begin{figure}[h]
\centering
\includegraphics[scale=0.65]{./data/diode/bias-rev.pdf}
\caption{逆バイアスの$V_{D}-I_{D}$特性}
\label{fig:bias-rev}
\end{figure}

\clearpage
\subsubsection{ダイオードの考察}
\begin{enumerate}[(1)]
\item 直流電圧電流特性グラフを説明せよ.(立ち上がり電圧$V_{J}$をグラフに書き込む.また,$E=0.8\,\rm{V}$としたときの負荷線を描き,動作点Pの微分抵抗$r_{d}=\Delta V_{D}/\Delta I_{D}$を求めよ.$r_{d}$,$V_{J}$,理想ダイオードからなる等価回路を描き,どの部分がどのような特性を表しているのか説明せよ)

\wfig{vias-graph-n}に$V_{J}$,負荷線,を加えたグラフを\wfig{vias-graph-n-1}に示す.
また,各特性値の導出方法を\wtab{how}にまとめる.

\begin{table}[h]
\centering
\caption{特性値の導出方法}
\label{tab:how}
\scalebox{0.8}{
\begin{tabular}{cl}
\hline
特性値    & 導出方法  \\
\hline
動作点$P$ &負荷線の式(\weq{IF})に電圧$V_{D}$を代入した$I_{D}'$と計測電流$I_{D}$の誤差が最も少ない点($0.7\,\rm{V}$, $19.8\,\rm{mA}$)とした.(\wtab{PT}) \\
接線    & 動作点$P$を中間点とするような$(0.675\,\rm{V}, 10\,\rm{mA})$と$(0.725\,\rm{V}, 29\,\rm{mA})$を変位として\weq{int}のように導出 \\
$V_J$   & 接線と$x$軸との交点を通る$y$軸に並行な直線とした.(\weq{vj}) \\
負荷線   & \weq{IF}のように合成抵抗と電圧値により導出\cite{1130282271098203264}.\\
微分抵抗$r_d$ & 動作点$P$を中間点とするような$(0.675\,\rm{V}, 10\,\rm{mA})$と$(0.725\,\rm{V}, 29\,\rm{mA})$を変位として\weq{rd}のように算出\\
\hline
\end{tabular}
}
\end{table}

\begin{align}
\centering
I_{D}&=\frac{\Delta I_{D}}{\Delta V_{D}} V_{D} +V_{J}\nonumber \\
\frac{\Delta I_{D}}{\Delta V_{D}} &=\frac{(29-10)\times 10^{-3}}{0.725-0.675}=380\times 10^{-3}\,\rm{S}\nonumber \\
V_{J}&=I_{D}-\frac{\Delta I_{D}}{\Delta V_{D}} V_{D}\nonumber \\
この&接線は動作点を通るので\nonumber \\
&=19.8\times 10^{-3}-380\times 10^{-3}\cdot 0.7=-246.2\,\rm{mV}\nonumber \\
\therefore I_{D}&=380V_{D} -246.2[\rm{mA}]\label{eq:int}
\end{align}
\begin{align}
\centering
V_{J}の座標&は接線でI_{D}=0となる点であるから\weq{int}より\nonumber \\
0&=380V_{J} -246.2\nonumber \\
246.2&=380V_{J}\nonumber \\
\therefore V_{J}& \fallingdotseq 0.65\,\rm{V} \label{eq:vj}
\end{align}
\begin{align}
\centering
R&=\frac{10\cdot10}{10+10}\nonumber \\
&=5\,\rm{\Omega}\nonumber \\
I_{D}&=-\frac{1}{R}V_{D}+\frac{E}{R}\nonumber \\
&=-\frac{1}{5}V_{D}+\frac{0.8}{5}\,\rm{A}\nonumber \\
&=\left(-\frac{1}{5}V_{D}+\frac{0.8}{5}\right)\times 10^{3}[\rm{mA}]
\label{eq:IF}
\end{align}

\begin{table}[h]
\centering
\caption{動作点$P$の導出}
\label{tab:PT}
\scalebox{1.0}{
\begin{tabular}{cccc}
\hline
電圧$V_{D}$[\rm{V}]  &計測電流$I_{D}$[\rm{mA}]   & 算出した$I_{D}'$[\rm{mA}] & 誤差$|I_{D}-I_{D}'|$[\rm{mA}] \\
\hline
0.000 & 0.0   & 160 & 160 \\
0.100 & 0.0   & 140 & 140 \\
0.200 & 0.0   & 120 & 120 \\
0.300 & 0.0   & 100 & 100 \\
0.400 & 0.0   & 80  & 80  \\
0.500 & 1.0   & 60  & 59  \\
0.600 & 2.0   & 40  & 38  \\
0.625 & 4.0   & 35  & 31  \\
0.650 & 6.0   & 30  & 24  \\
0.675 & 10.0  & 25  & 15  \\
0.700 & 19.8  & 20  & 0.2 \\
0.725 & 29.0  & 15  & 14  \\
0.750 & 48.0  & 10  & 38  \\
0.775 & 71.0  & 5   & 66  \\
0.800 & 116.0 & 0   & 116 \\
\hline
\end{tabular}
}
\end{table}

\begin{figure}[h]
\centering
\includegraphics[scale=0.65]{./data/diode/bias-n-1.pdf}
\caption{順バイアスにおける負荷線と$V_{J}$}
\label{fig:vias-graph-n-1}
\end{figure}
\begin{align}
r_{d}&=\frac{\Delta V_{D}}{\Delta I_{D}}\nonumber \\
&=\frac{0.725-0.675}{(29.0-10.0)\times 10^{-3}}\nonumber \\
&=\frac{0.05}{19.0\times 10^{-3}}\nonumber \\
& \fallingdotseq 2.63\,\rm{\Omega}
\label{eq:rd}
\end{align}

理想ダイオードからなる等価回路は\wfig{eqc}となる~\cite{adsfcaw}.
$V_{J}$はダイオードの内部で発生する電界から生まれる電圧降下を示し,
$r_{d}$は動作点近傍でのダイオードの抵抗を表す~\cite{sdfvadfcdf}.
理想ダイオードは実際のダイオードと異なり,順方向にバイアスされているまたは,端子電圧が$0\,\rm{V}$の際に完全な導体として振る舞い,逆バイアスが印加されている時に完全な絶縁体となるものである.これらがダイオードの整流作用を起こしている~\cite{adsfcaw}.
	\begin{figure}[h]
	\centering
	\includegraphics[scale=0.5]{./fig/eqc.pdf}
	\caption{理想ダイオードからなる等価回路}
	\label{fig:eqc}
	\end{figure}

	\item 実験で用いたダイオードのデータシートから逆方向電流の値を調べよ.また,逆バイアスの場合でもわずかに電流が流れる理由を図を用いて説明せよ.(多数キャリアや少数キャリアという観点から考えること。\wfig{fig2}を用いるとよい.)

データシート~\cite{sdfgvhsd}より逆電流$I_{R}$は$25^{\circ}$Cで最大$10\,\rm{\mu A}$流れることがわかる.
逆バイアスを印加した際,接合面で多数キャリアが再結合で減少する.
また\wfig{fig2}ではp型において,多数キャリアのみに注目しているため,電子が移動せず,電流が流れないように考えられるが,実際は少数キャリアである電子が移動する(ドリフト)ためわずかに電流が流れてしまう.
\end{enumerate}

\clearpage
\subsection{ツェナーダイオードの特性実験}
\subsubsection{ツェナーダイオードの実験器具}
使用した実験器具を\wtab{kigu2}に示す.
\begin{table}[h]
  \centering
  \caption{実験装置}
  \label{tab:kigu2}
  \scalebox{1.0}{
  \begin{tabular}{cccccc}
    \hline
    機器名&製造元&型番&シリアル番号(または管理番号)\\
    \hline
    ダイオード&不明&1N4736A&不明\\
    直流電源&YOKOGAWA&PA1811&L96-000668\\
    直流用電圧計&YOKOGAWA&YAS 1991&71 BA0 3371\\
    ミリアンペア直流用電流計&YOKOGAWA&YES 1990&70 BA0 1812\\
    \hline
  \end{tabular}
}
\end{table}

\subsubsection{ツェナーダイオードの実験方法}
\begin{enumerate}[(1)]
\item \wfig{zenerc}のように回路を構築する.なお,電圧計は$10\,\rm{V}$端子,電流計は$30\,\rm{mA}$端子に接続して計測を行った.また,配線時にダイオードの向きに注意する必要がある.
\item $V_{i}$を$0\,\rm{V}$から$18\,\rm{V}$まで$0.1\,\rm{V}$刻みで増加させ,電流$I_{Z}$と電圧$V_{L}$を計測する.
\item 上記で計測したデータをもとに,$V_{i}-V_{L}$特性,$V_{i}-I_{Z}$を同一グラフに描画した.
\begin{figure}[h]
\centering
\includegraphics[scale=0.7]{./fig/zenerc.pdf}
\caption{ツェナーダイオード測定回路}
\label{fig:zenerc}
\end{figure}
\end{enumerate}

\subsubsection{ツェナーダイオードの結果}
\begin{itemize}
\item \wtab{zener-tab}に測定したデータをまとめた.
出力電圧が$10\,\rm{V}$付近からツェナー電流の増加が起きていることがわかる.
また,出力電圧$V_{L}$は$8\,\rm{V}$以降,入力電圧が増加してもあまり変化が起きていない.
\item \wfig{zenerg}は$V_{i}-V_{L}$特性,$V_{i}-I_{Z}$特性のグラフである.$V_{L}$は増加した後,ほぼ一定となっているのに対し,$I_{Z}$はほぼ一定($=0$)の後,増加と逆の動きをしている.
\begin{table}[h]
\centering
\caption{ツェナーダイオードの特性}
\label{tab:zener-tab}
\scalebox{0.9}{
\begin{tabular}{ccc}
\hline
入力電圧$V_i$$[\rm{V}]$ & 出力電圧$V_L$$[\rm{V}]$ & ツェナー電流$I_Z$$[\rm{mA}]$ \\
\hline
0    & 0.0       & 0.0      \\
1    & 0.8     & 0.0       \\
2    & 1.5       & 0.0          \\
3    & 2.3         & 0.0              \\
4    & 3.0           & 0.0              \\
5    & 3.8         & 0.0              \\
6    & 4.6         & 0.0           \\
7    & 5.3         & 0.0              \\
8    & 6.1         & 0.0           \\
9    & 6.8         & 0.1       \\
10   & 6.9         & 3.0           \\
11   & 6.9         & 6.2          \\
12   & 7.0           & 9.6           \\
13   & 7.0           & 12.8           \\
14   & 7.0           & 16.1           \\
15   & 7.0           & 19.4           \\
16   & 7.1         & 22.6        \\
17   & 7.1         & 25.9           \\
18   & 7.2      & 29.2     \\
\hline
\end{tabular}
}
\end{table}
\begin{figure}[h]
\centering
\includegraphics[scale=0.65]{./data/zener/zener-graph.pdf}
\caption{ツェナーダイオードの特性}
\label{fig:zenerg}
\end{figure}
\end{itemize}

\clearpage
\subsubsection{ツェナーダイオードの考察}
\begin{enumerate}[(1)]
\item 今回用いたツェナーダイオード1N4736Aをダイオード規格表(または,データシート)でツェナー電圧を調べ,実験結果と比較して考察せよ.

データシート~\cite{fbklsdn}を参照すると平均値は$6.8\,\rm{V}$($25^{\circ}$C)である.一方,実験結果より,ツェナー電圧は$6.8\,\rm{V}$から$7.2\,\rm{V}$で推移しており,データシートとの差異がほとんどなく,正しく計測できたといえる.
\item ツェナーダイオードはどういった場合に用いられるか説明せよ.

ツェナーダイオードは\wfig{diode-vi-curve-03}からわかるように,逆電圧を一定(降伏電圧, $V_{R}$)以上かけると急激に電流が流れるようになる特性がある.
この降伏電圧付近では,電流の広い範囲にわたって電圧が一定である.
すなわち,ダイオードに流れる電流の大きさに依存せずダイオードの電圧は一定に保たれるため.定電圧源として利用される\cite{1130282271098203vsdv4}.
\begin{figure}[h]
\centering
\includegraphics[scale=0.15]{./fig/diode-vi-curve-03.png}
\caption{ツェナーダイオードの特性図\cite{sdjabcklds}}
\label{fig:diode-vi-curve-03}
\end{figure}
\item  $V_{Z}-I_{Z}$特性のグラフを描き,ツェナー電圧$V_{Z}$を明記せよ.

\wfig{zener-v-i}に$V_{Z}-I_{Z}$特性を示す.なお,$V_{Z}$は$6.8\,\rm{V}$とした.
出力電圧はほぼツェナー電圧に近似できているが,実験結果と誤差が少し発生している.
\begin{figure}[h]
\centering
\includegraphics[scale=0.7]{./data/zener/zener-v-i.pdf}
\caption{ツェナーダイオードの特性}
\label{fig:zener-v-i}
\end{figure}
\item 各グラフから入出力特性を考察せよ.

入力特性($V_{i}-I_{Z}$)について考える.\\
\wfig{zenerg}より,立ち上がり電圧$V_{F}$は$0.9\,\rm{V}$で,以降$I_{Z}$は$V_{i}$とほぼ比例して増加している.その傾きは約$33\times 10^{-3}\,\rm{S}$である.\\
次に出力特性($V_{L}-I_{Z}$)について考える.\\
\wfig{zener-v-i}より,$V_{Z}$以降,急激に電流を通すようになる.そしてその時の電圧は$6.8\,\rm{V}$から$7.3\,\rm{V}$に落ちついている.
\end{enumerate}

\clearpage
\subsection{ツェナーダイオード定電圧回路の実験}
\subsubsection{ツェナーダイオード定電圧回路の実験器具}
使用した実験器具を\wtab{kigu3}に示す.
\begin{table}[h]
  \centering
  \caption{実験装置}
  \label{tab:kigu3}
  \scalebox{0.9}{
  \begin{tabular}{cccccc}
	\hline
	機器名&製造元&型番&シリアル番号(または管理番号)\\
	\hline
	ダイオード&不明&1N4736A&不明\\
	直流電源&YOKOGAWA&PA1811&L96-000668\\
	直流用電圧計&YOKOGAWA&YAS 1991&71 BA0 3371\\
	ミリアンペア直流用電流計&YOKOGAWA&YES 1990&70 BA0 1812\\
	可変抵抗器&TOKUSHU DENKI KOGYOSHO&S-3&3201\\
	\hline
  \end{tabular}
}
\end{table}

\subsubsection{ツェナーダイオード定電圧回路の実験方法}
\begin{enumerate}[(1)]
	\item \wfig{zenerc-2}のように回路を構築する.なお,ダイオード$ZD$は1N4736Aを使用した.
\begin{figure}[h]
	\centering
	\includegraphics[scale=0.65]{./fig/zenerc-2.pdf}
	\caption{ツェナーダイオード定電圧測定回路}
	\label{fig:zenerc-2}
\end{figure}
	\item 入力$V_i$を$15\,\rm{V}$に固定し,可変抵抗$R_{L}$を変化させてツェナー電流$I_Z$を$2\,\rm{mA}$刻みで$2\,\rm{mA}$から$22\,\rm{mA}$まで変化させた.その際,電流計は端子$30\,\rm{mA}$に接続した.
	\item それぞれの場合で,出力電圧$V_L$と,出力電流$I_Z$を計測した.
	\item 上記のデータを元に$I_{Z}-V_{Z}$,$I_Z-I_L$のグラフを作成した.
\end{enumerate}

\subsubsection{ツェナーダイオード定電圧回路の結果}
\begin{itemize}
	\item \wfig{zener-const-1}に$I_{Z}-V_{Z}$,\wfig{zener-const-2}に$I_{Z}-I_{L}$特性グラフを示す.
出力電圧$V_L$はツェナー電流$I_Z$に依存せず,$7\,\rm{V}$付近で一定であった.
また,出力電流$I_{L}$はツェナー電流$I_Z$増加とともにほぼ比例的に減少している.
\begin{figure}[h]
	\centering
	\includegraphics[scale=0.65]{./data/zener/zener-const-1.pdf}
	\caption{定電圧回路の$I_Z-V_Z$特性}
	\label{fig:zener-const-1}
\end{figure}
\begin{figure}[h]
	\centering
	\includegraphics[scale=0.65]{./data/zener/zener-const-2.pdf}
	\caption{定電圧回路の$I_Z-I_L$特性}
	\label{fig:zener-const-2}
\end{figure}
\end{itemize}

\clearpage
\subsubsection{ツェナーダイオード定電圧回路の考察}
\begin{enumerate}[(1)]
	\item 等価回路図(\wfig{zener-eq})を参考にして,測定結果から$R_{L}$,$R_{Z}$,およびこれらの合成抵抗$R_{A}$を算出せよ.
	
	\begin{figure}[h]
	\centering
	\includegraphics[scale=0.75]{./fig/zener-eq.pdf}
	\caption{ツェナーダイオード定電圧等価回路}
	\label{fig:zener-eq}
	\end{figure}
	\wfig{zener-eq}より,この回路は並列回路であることがわかり,分圧則とオーム則を用いて抵抗値が算出可能である.
	\weq{RL}から\weq{RA}に各抵抗値の導出方法を示す.
\begin{align}
	R_{L}&=\frac{V_{L}}{I_{L}}\label{eq:RL}\\
	R_{Z}&=\frac{V_{L}}{I_{Z}}\label{eq:RZ}\\
	R_{A}&=300+\frac{R_{Z}\cdot 1.0\times 10^{3}}{R_{Z}+1.0\times 10^{3}}\label{eq:RA}
\end{align}
	これらの計算を各ツェナー電流ごとに行ったものを\wtab{rtab}に示す.
	
\begin{table}[h]
\centering
\caption{電流,電圧値により算出した各抵抗値}
\label{tab:rtab}
\scalebox{1.0}{
\begin{tabular}{cccc}
	\hline
	ツェナー電流$I_{Z}$[$\rm{mA}$] & 可変抵抗値$R_{L}$[$\Omega$] & ダイオード抵抗値$R_{Z}$[$\Omega$] & 合成抵抗$R_{A}$[$\Omega$]  \\
	\hline
	 2     & 0.276 & 3.450    & 0.256 \\
	 4     & 0.300 & 1.725    & 0.256 \\
	 6     & 0.329 & 1.150    & 0.256 \\
	 8     & 0.367 & 0.863    & 0.257 \\
	10    & 0.419 & 0.700    & 0.262 \\
	12    & 0.476 & 0.583    & 0.262 \\
	14    & 0.556 & 0.500    & 0.263 \\
	16    & 0.676 & 0.444    & 0.268 \\
	18    & 0.845 & 0.394    & 0.269 \\
	20    & 1.127 & 0.355    & 0.270 \\
	22    & 1.690 & 0.323    & 0.271 \\
	\hline
\end{tabular}
}
\end{table}
この表より,合成抵抗$R_{A}$はほとんどツェナー電流に依存せずに一定の値をとっていることがわかる.このことは,\weq{ZL}で右辺の項が定数のため,左辺の電流値も定数となっているためである.
\begin{equation}
	I_{Z}+I_{L}=\frac{V_{i}-V_{L}}{300}
	\label{eq:ZL}
\end{equation}
\begin{figure}[h]
	\centering
	\includegraphics[scale=0.7]{./data/zener/r.pdf}
	\caption{各抵抗のツェナー電流特性}
	\label{fig:rgraph}
\end{figure}
	\item $I_{Z}-R_{L}$特性,$I_{Z}-R_{Z}$特性を同一グラフに重ねて描き,考察せよ.ここで,何について論じたいかを明確にした上で記述すること.

	\wfig{rgraph}は上記式により導出した各抵抗値のツェナー電流特性である.
	合成抵抗はツェナー電流の変化に依存していないが,$R_{L}$は$I_{Z}$に比例し,$R_{Z}$は反比例していることが読み取れる.これは上の考察と矛盾がない.
\end{enumerate}
\clearpage
\subsection{太陽電池の発電特性の実験}

\subsubsection{太陽電池の発電特性の実験器具}
使用した実験器具を\wtab{kigu4}に示す.
\begin{table}[h]
  \centering
  \caption{実験装置}
  \label{tab:kigu4}
  \scalebox{0.7}{
  \begin{tabular}{cccccc}
	\hline
	機器名&製造元&型番&シリアル番号(または管理番号)\\
	\hline
	太陽電池&SUNYO&SY-M5W&-\\
	ディジタル照度計& Zhangzhou Weihua Electronic Co., Ltd&LX-1010B&T 428585\\
	ディジタル温度計&Gain Express Holdings&THE-27 Digital Thermometer 4 Channel K-Type&202019031\\
	ディジタルマルチメータ&owon&B35&B351518443\\
	可変抵抗器&TOKUSHU DENKI KOGYOSHO&S-3&3201\\
	\hline
  \end{tabular}
}
\end{table}

\subsubsection{太陽電池の発電特性の実験方法}
\begin{enumerate}[(1)]
	\item \wfig{solar-cell}のように回路を構築する.
	\begin{figure}[h]
	\centering
	\includegraphics[scale=0.5]{./fig/solar-cell.pdf}
	\caption{太陽電池の計測回路}
	\label{fig:solar-cell}
	\end{figure}
	\item 太陽電池はライトの光が均等に照射されるように配置した.
	\item 可変抵抗器のレバーを短絡側にセットし,ライトを2つ点灯させた.
	\item 太陽電池を動かし,電流が最大になる位置を探し位置に印をつけた.
	\item ライトを1つ点灯に変更した.
	\item 太陽電池の四隅と中央の計5ヶ所に関して計測を行い.その平均値を代表値として,算出した.
	\item 太陽電池の裏面中央に温度計を設置した.
	\item 照度測定後,太陽電池温度が$30^{\circ}$Cになるように調節を行った.\label{ondo}
	\item $V-I$測定を開始した.可変抵抗$R_{L}$を変更させながら,電流$I$,電圧$V$を記録する.$V_{oc}$付近ではデータ取得間隔を細かくした.\label{I-V}
	\item 測定の際温度は$\pm 2^\circ$Cの範囲で計測を行った.
	\item 点灯させるライトを2つに変更し,上記と同様((\ref{ondo}) $\sim$ (\ref{I-V}))に計測を行った.
	\item 太陽電池温度を$50^{\circ}$Cに変更し,点灯させるライトを1つ,2つの場合に対し測定を行った.
\end{enumerate}

\subsubsection{太陽電池の発電特性の結果}
\label{solar-cell-result}
\begin{itemize}
	\item $30^{\circ}$C,ライト1つ・2つ,$50^{\circ}$C,ライト1つ・2つの計4条件での計測結果を\wtab{30-one}$\sim$\wtab{50-two}に示す.
	上記データをもとに作成した,$V-I$カーブを\wfig{VIc}に示す.
	
	これらより電流が多く流れることに関わる因子の影響は,太陽電池温度ではなく,ライトの個数(照度)の方が大きいことがわかる.また,出力される電流の最大値は決まっており,ある電圧を超えるとほぼ一定であった電流値が減少に変化することが確認される.
	そして,その電圧の値は4条件で大きく違いがなかった.
	\item \weq{power}をExcelを用いて算出した電力を用いて作成した$V-P$カーブを\wfig{VPc}に示す.
	\begin{equation}
		P=VI
		\label{eq:power}
	\end{equation}
	\begin{table}[p]
	\begin{tabular}{cc}
	\begin{minipage}[t]{0.5\hsize}
	\centering
	\caption{$30^{\circ}$C ライト1つの場合(照度8600\,\rm{lux})}
	\label{tab:30-one}
	\scalebox{1.0}{
	\begin{tabular}{ccc}
	\hline
	電圧$V$[\rm{V}] & 電流$I$[\rm{mA}]   & 電力$P$[\rm{W}] \\
	\hline
	0.8  & 63.8 & 0.1 \\
	2.8  & 63.7 & 0.2 \\
	3.5  & 63.0 & 0.2 \\
	7.0  & 63.1 & 0.4 \\
	9.8  & 62.6 & 0.6 \\
	12.3 & 62.3 & 0.8 \\
	14.3 & 61.9 & 0.9 \\
	16.7 & 61.5 & 1.0 \\
	17.3 & 60.0 & 1.0 \\
	18.3 & 56.5 & 1.0 \\
	18.7 & 50.2 & 0.9 \\
	18.9 & 45.3 & 0.9 \\
	19.2 & 40.5 & 0.8 \\
	19.4 & 35.3 & 0.7 \\
	19.5 & 30.7 & 0.6 \\
	19.6 & 25.7 & 0.5 \\
	20.0 & 20.5 & 0.4 \\
	20.1 & 15.5 & 0.3 \\
	20.1 & 10.6 & 0.2 \\
	20.2 & 6.1  & 0.1 \\
	\hline
	\end{tabular}
	}
	\end{minipage}
	\begin{minipage}[t]{0.5\hsize}
	\centering
	\caption{$30^{\circ}$C ライト2つの場合}
	\label{tab:30-two}
	\scalebox{0.9}{
	\begin{tabular}{ccc}
	\hline
	電圧$V$[\rm{V}]   & 電流$I$[\rm{mA}]   & 電力$P$[\rm{W}]  \\
	\hline
	1.9  & 151.1 & 0.3 \\
	2.7  & 151.0 & 0.4 \\
	3.9  & 148.8 & 0.6 \\
	5.6  & 148.3 & 0.8 \\
	7.1  & 147.2 & 1.0 \\
	9.9  & 144.0 & 1.4 \\
	10.7 & 141.5 & 1.5 \\
	15.9 & 137.6 & 2.2 \\
	13.7 & 141.0 & 1.9 \\
	18.6 & 136.4 & 2.5 \\
	19.3 & 127.5 & 2.5 \\
	19.7 & 105.8 & 2.1 \\
	20.0 & 91.2  & 1.8 \\
	20.4 & 80.3  & 1.6 \\
	20.2 & 74.5  & 1.5 \\
	20.3 & 61.8  & 1.3 \\
	20.8 & 50.6  & 1.1 \\
	20.9 & 40.1  & 0.8 \\
	20.8 & 29.3  & 0.6 \\
	20.9 & 20.0  & 0.4 \\
	21.2 & 10.6  & 0.2 \\
	21.1 & 6.4   & 0.1 \\
	\hline
	\end{tabular}
	}
	\end{minipage}\\
	\begin{minipage}[t]{0.5\hsize}
	\centering
	\caption{$50^{\circ}$C ライト1つの場合}
	\label{tab:50-one}
	\scalebox{0.75}{
	\begin{tabular}{ccc}
	\hline
	電圧$V$[\rm{V}]   & 電流$I$[\rm{mA}]   & 電力$P$[\rm{W}]  \\
	\hline
	1.1  & 65.0 & 0.1 \\
	2.7  & 64.6 & 0.2 \\
	5.0  & 64.6 & 0.3 \\
	6.8  & 64.1 & 0.4 \\
	7.1  & 64.3 & 0.5 \\
	9.5  & 64.0 & 0.6 \\
	12.0 & 63.5 & 0.8 \\
	14.2 & 63.1 & 0.9 \\
	16.5 & 61.6 & 1.0 \\
	16.5 & 60.3 & 1.0 \\
	17.3 & 53.3 & 0.9 \\
	17.8 & 50.5 & 0.9 \\
	18.2 & 39.1 & 0.7 \\
	18.4 & 34.9 & 0.6 \\
	18.5 & 30.2 & 0.6 \\
	18.8 & 20.4 & 0.4 \\
	19.0 & 16.1 & 0.3 \\
	18.9 & 15.1 & 0.3 \\
	19.0 & 14.1 & 0.3 \\
	19.0 & 13.1 & 0.2 \\
	19.0 & 11.1 & 0.2 \\
	19.0 & 10.2 & 0.2 \\
	19.1 & 9.1  & 0.2 \\
	19.1 & 7.1  & 0.1 \\
	19.2 & 5.8  & 0.1 \\
	\hline
	\end{tabular}
	}
	\end{minipage}
	\begin{minipage}[t]{0.5\hsize}
	\centering
	\caption{$50^{\circ}$C ライト2つの場合}
	\label{tab:50-two}
	\scalebox{1.0}{
	\begin{tabular}{ccc}
	\hline
	電圧$V$[\rm{V}]   & 電流$I$[\rm{mA}]   & 電力$P$[\rm{W}]  \\
	\hline
	2.6  & 151.4 & 0.4 \\
	5.5  & 149.8 & 0.8 \\
	7.5  & 147.9 & 1.1 \\
	10.6 & 146.5 & 1.6 \\
	13.0 & 144.0 & 1.9 \\
	15.3 & 143.7 & 2.2 \\
	17.3 & 142.0 & 2.5 \\
	18.3 & 134.4 & 2.5 \\
	18.4 & 129.3 & 2.4 \\
	18.8 & 103.1 & 1.9 \\
	19.4 & 80.4  & 1.6 \\
	19.7 & 58.8  & 1.2 \\
	19.8 & 41.0  & 0.8 \\
	19.9 & 30.4  & 0.6 \\
	20.0 & 20.7  & 0.4 \\
	20.0 & 11.6  & 0.2 \\
	20.2 & 6.1   & 0.1 \\
	\hline
	\end{tabular}
	}
	\end{minipage}
	\end{tabular}
	\end{table}
	\begin{figure}[h]
	\begin{minipage}[c]{1.0\hsize}
	\centering
	\includegraphics[scale=0.7]{./data/solar-cell/solar-cell.pdf}
	\caption{太陽電池の$V-I$カーブ}
	\label{fig:VIc}
	\end{minipage}
	\begin{minipage}{1.0\hsize}
	\centering
	\includegraphics[scale=0.7]{./data/solar-cell/solar-cell-2.pdf}
	\caption{太陽電池の$V-P$カーブ}
	\label{fig:VPc}
	\end{minipage}
	\end{figure}
\end{itemize}

\clearpage
\subsubsection{太陽電池の発電特性の考察}
\begin{enumerate}[(1)]
	\item 各条件下の電流-電圧の測定結果から電力を算出し,$V-P$カーブを描きなさい.また,最大電力点を抽出して表にまとめ,条件の変化と最大電力点について考察せよ.
	
	$V-P$カーブは\wfig{VPc}に示した.\\
	最後に最大電流値を取る点が電力の最大点とほぼ一致していることが読み取れる.
	また,電力最大点より,小さい電圧の電流の傾きと.大きい電圧の電流の傾きは異なり,前者は緩やかで,後者は急である.
	電圧がある一定値を超えると急に電力を得ることが難しくなるとうことである.\\
	また,条件を変更しても電力最大点の移動が確認されなかったため,この点は太陽電池固有のものであると考えられる.$30^{\circ}$Cでライト2の場合つまり,低温で高い照度の場合が最も電力を多く得ることができている.\\
	電力は電流に比例し,\ref{solar-cell-result}で述べたように,より多くの電流ためにはより高い照度が必要であり,高温になると,低温時より,電力の低下がみられる.\\
	pn接合ダイオード電流は\weq{exp}で電流が算出できる.この式より,温度が増加すると電流が減少することが読み取れるため,上記の考察は正しいと考えられる.
	\item MPPT制御はなぜ必要か説明せよ.また,そのアルゴリズムには「山登り法」と「電圧追従法」がある.これらの特徴を調査し,比較して説明せよ.
	
	\wfig{VPc}の最大電力点で発電を行うように発電システムを構築したとしても,天候などの要因により電圧の増減し,電力が減少してしまう.そのため,最大電力点を用いた制御方法が必要となる.
	\begin{enumerate}
		\item[山登り法:] \wfig{mountan}の$P-V$曲線で電圧を一方向(増加または減少)に変化させていき,電力が増加から減少に転換すると電圧を変化させる方向を逆方向にする. これを繰り返すことにより,常に電力が最大となる最適動作点に制御する方法である\cite{esfvjsp}.
		最大電力出力点が山の頂に見え,山に登っていくように見えることからこの名前がついている.
		しかし,次に述べる電圧追従法より回路・制御が複雑であるため,コストや消費電力が高いという負の面もある\cite{nvdfsjknv}.
		\begin{figure}[h]
		\centering
		\includegraphics[scale=0.5]{./fig/mountan.png}
		\caption{山登り法\cite{esfvjsp}}
		\label{fig:mountan}
		\end{figure}
		\item[電圧追従法:]$V-I$出力特性を利用する方法である.\wfig{area}の枠の面積が発電電力を表し,$V1$のように発電電流が高く,$V2$のように発電電力が高くてもバランスが悪く,効率よく発電が行えていない状態である.しかし,$Vm$のようにそれぞれのバランスが良い状態の時,太陽電池は最大電力で発電が可能になる.この時の電圧値は開放電圧の約$80\,\%$である.
		太陽電池の動作点がバッテリーや負荷の電圧の影響を受けないため,一定の異常の効率で太陽電池から電力を取り出すことが可能である.また,電圧値の計測で制御が行えるため,容易で比較的安価であるが,気象条件により最適電圧が変動するため,最適電圧で動作していると言えない場合がある\cite{nvdfsjknv}.
		\begin{figure}[h]
		\centering
		\includegraphics[scale=1]{./fig/area.png}
		\caption{電圧追従法\cite{nvdfsjknv}}
		\label{fig:area}
		\end{figure}
		\end{enumerate}
		なお,現在主流である制御方法は山登り法である\cite{main}.
\end{enumerate}

\clearpage
\section{結論}
この実験を通して,以下の点を達成することができた.
\begin{itemize}
	\item ダイオードの特性の1つである整流特性を実験により確認し,ダイオードを使用する
	\item ツェナーダイオード特性を理解し,利用する.
	\item 太陽電池の利用環境特性を知り,活用する.
\end{itemize}

\newpage
\printbibliography[title=参考文献]
\end{document}

